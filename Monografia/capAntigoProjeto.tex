\postextualchapter{Projeto Original}

\section{Introdução}
O \textbf{Museu D. João VI da Escola de Belas Artes da Universidade Federal do Rio de Janeiro} foi criado em 1979 com a finalidade de preservar a memória do ensino artístico oficial e de fomentar o estudo e a pesquisa da História da Arte Brasileira. Ele vem responder a necessidade da criação de um espaço institucional de preservação do património e memória do ensino de arte, reunindo a produção da Academia Imperial de Belas Artes, da Escola Nacional de Belas Artes e parte da história recente da Escola de Belas Artes.

O Museu abriga dois acervos distintos, um de obras de arte e outro de documentos, fontes primárias indispensáveis para o desenvolvimento de estudos e projetos de pesquisa em arte, quer no campo teórico quer no aplicado. Estes acervos são o resultado do patrimônio acadêmico produzido pela Escola no período compreendido, principalmente, entre 1820 e 1920.

Com o objetivo de implementar a atividade de pesquisa foi desenvolvido um projeto integrado - Museu D. João VI e Mestrado em História da Arte - EBA/UFRJ, financiado pelo CNPq, que tratou de realizar o inventário científico e sistemático destes dois acervos. 

O processo de sistematização culminou com a sua informatização, por meio de um convênio com o Núcleo de Computação Eletrônica da UFRJ, sendo o Núcleo responsável pela montagem global do sistema.

O trabalho de catalogação e sistematização foi elaborado de forma o mais técnica e criteriosa possível, sendo aplicados inúmeros instrumentos nas suas diversas etapas de identificação e de realização dos inventários. A descrição destes instrumentos, as metodologias aplicadas e os critérios de seleção e ordenação dos acervos são apresentados a seguir, se constituindo em documento técnico imprescindível, não só para o entendimento do processo, assim como para a ampliação da base do banco de dados.

\section{ACERVO MUSEOLÓGICO}
O acervo de obras do \textbf{Museu D. João VI (MDJVI)} tem uma importância singular seja para o estudo e o entendimento da história da formação artística no país, seja para a construção de uma história da arte brasileira. O acervo original da Academia Imperial de Belas Artes tem sua origem na contratação da \textbf{Missão Artística Francesa}, que chegando ao Brasil trouxe na bagagem uma coleção de 42 obras de artistas franceses e italianos, inaugurando a pinacoteca da antiga Academia. Soma-se ainda um número significativo de moldagens de gesso destinadas ao Curso de Escultura, servindo até hoje como material didático. 

Outras coleções foram sendo incorporadas à Academia e à Escola de Belas Artes, tais como as coleções de gravuras holandesas e francesas. Além destas encontra-se uma extensa coleção de medalhas produzidas pelo antigo Curso de Gravura de medalhas e pedras preciosas e, uma importante coleção de desenhos, formada pelos trabalhos de alunos e professores, compreendendo os Cursos de Desenho artístico, Modelo vivo e Desenho arquitetônico. O acervo se caracteriza por reunir um volume considerável de obras-documento, apresentadas nos concursos para professor, como aquelas referentes aos concursos para Prêmio de viagem. Destacam-se, ainda, inúmeras obras selecionadas e premiadas nas exposições oficiais da Academia, entre elas as \textit{Exposições Gerais de Belas Artes} (Salão).

O trabalho de sistematização do acervo museológico compreendeu duas etapas: a realização do inventário completo, com a identificação e fichamento das obras, e a informatização e criação do banco de dados.
Na primeira etapa foi feito um levantamento minucioso da fonte primária, que compreende o acervo completo de artes visuais - desenhos, gravuras, esculturas e pinturas - tomando como base o \textbf{Livro de Tombo do Museu D. João VI} Este processo incluiu a revisão completa das peças já inventariadas, chamando a atenção para duas coleções não museológicas registradas no \textbf{Livro de Tombo}, que são: diplomas e fotografias.

Para a etapa de informatização deste \textbf{Inventário Geral} foi elaborada uma ficha de registro de obras de acordo com as normas técnicas atuais, partindo de dados básicos e universais no que concerne à coleção de artes visuais, apresentando os seguintes campos, distribuídos em duas categorias de informação: dados gerais e dados complementares.

\subsection{Dados Gerais}
\subsubsection{Número de registro}
É o \textbf{número de registro de tombamento}. A numeração respeitou parcialmente o sistema numérico do \textbf{Livro de Tombo} encontrado, seguindo portanto, a sequência natural do número de ordem deste livro, cuja última peça fora registrada sob o número \textbf{2140}. A primeira peça sem registro a ser inventariada recebeu, consequentemente, o número 2141.Para tornar o sistema de numeração mais objetivo, optou-se por manter somente o número de ordem, abandonando o modelo tripartido do livro de tombo original, ou seja, omitindo- se os dígitos referentes ao ano de entrada da peça e à sua categoria. Para as peças compostas de diversas partes adotou-se o uso do complemento alfabético: 
\begin{itemize}
	\item registro \textbf{1463 A} - Classe Objetos pessoais, Subclasse Objetos de adorno, Relógio - e 
	\item registro \textbf{1463 B} - Classe Objetos pessoais, Subclasse Objetos de
	adorno, Estojo.
\end{itemize}

\subsubsection{Classe e Subclasse}
Os conceitos de classe e subclasse adotados foram baseados no ``\textit{Thesaurus para Acervos Museológicos}'' - Fundação Pró-Memória/1987, utilizado, na atualidade, por inúmeros museus brasileiros. De acordo com este sistema a classificação obedece a um único critério, o da funcionalidade original da peça, abolindo-se as classificações antigas, que combinavam a função com materiais e categorias. 

No caso específico do \textbf{MDJVI} a classe predominante no acervo é \textbf{Artes visuais} com suas respectivas sub-classes: 
\begin{itemize}
	\item Construção artística, 
	\item Desenho, 
	\item Desenho arquitetônico, 
	\item Escultura, 
	\item Gravura e 
	\item Pintura.
\end{itemize} 

As outras classes e sub-classes foram também empregadas, mas com menos frequência em função da própria especificidade do acervo em questão que são elas, classes: 
\begin{itemize}
	\item Amostras/fragmentos, 
	\item Caça/guerra, 
	\item Comunicação, 
	\item Construção, 
	\item Embalagem/recipientes, 
	\item Insígnia, 
	\item Interiores, 
	\item Lazer/desporto,
	\item Medição/registro/observação/processamento, 
	\item Objetos cerimoniais, 
	\item Objetos pessoais e 
	\item Trabalho.
\end{itemize} 

Como norma ortográfica para as Classe e Subclasses
estipulou-se o uso de maiúscula somente na primeira letra das nomenclaturas: \textit{Classe - Artes visuais, Subclasse - Desenho arquitetônico.}

As coleções de diploma e fotografia constituíram-se em caso especial pois foram registradas no \textbf{Livro de Tombo do Museu D. João VI}, segundo a classificação prevista no \textit{Thesaurus para acervos museológicos} que prevê o uso da classe \textbf{Comunicação} para os objetos usados para transmitir informações aos seres humanos\footnote{FERREZ Helena Dodd e BIANCHINI, Maria Helena S. \textit{Thesaurus para acervos museológicos}. p. 7.}, e a subclasse \textbf{Documento} para documentos textuais, cartográficos e iconográficos. Constituindo-se, seu uso, caso específico para museus que não possuem setor técnico para tratamento deste tipo de acervo. Baseado nesta particularidade, optou-se pela continuidade do registro estabelecendo-se adaptações para o preenchimento dos campos da ficha de registro. Como norma técnica para este campo, utilizou-se o \textit{Termo} ou \textit{Nome do objeto} junto à subclasse para evitar dúvidas quanto ao acervo consultado, criando-se as classificações: \textit{Comunicação - Documento diploma e Comunicação - Documento fotográfico}.


\subsubsection{Autoria}
Campo destinado ao registro de pessoa física ou jurídica que concebeu material e/ou intelectualmente a obra, acompanhado de cronologia. O autor foi situado cronologicamente tendo como referência os \textbf{anos de nascimento e morte}, com informações coletadas do \textit{Dicionário das Artes Plásticas no Brasil}, \textit{Dicionário Brasileiro de Artistas Plásticos} e \textit{Dictionnaire des Peintres, Sculpteurs, Dessinateurs et Graveurs}, citados em referência bibliográfica do projeto em questão. Para a normatização dos dados estipulou-se:

\begin{itemize}
	\item para entrada de dados no campo registrar-se em maiúscula o último sobrenome do autor, seguido de vírgula, do prenome ou prenomes em minúscula, e dos anos de nascimento e morte entre parênteses.
	\item para o caso de autores não identificados registrar-se no campo, não identificada;
	\item para pseudônimos registrá-los entre aspas, precedidos do autor e dos anos de nascimento e morte entre parênteses;
	\item para a coleção de diplomas registrar-se a instituição ou órgão gerador do documento em maiúsculas;
	\item para a coleção de gravuras registrar-se o autor que elaborou a matriz com o último sobrenome em maiúscula seguido de vírgula, do prenome ou prenomes em minúscula e os anos de nascimento e morte entre parênteses.
\end{itemize}

\subsubsection{Assinatura}
Destina-se ao registro do local da assinatura do autor na obra. A informação sobre a localização da assinatura na obra seguiu as normas do \textit{Manual de Catalogação de Pinturas, Esculturas, Desenhos e Gravuras}, publicado pelo \textbf{Museu Nacional de Belas Artes}. Listagem em anexo.

\subsubsection{Co-autoria}
Identifica os demais autores envolvidos na criação da obra, especificamente para as subclasses de \textbf{gravura} e \textbf{diploma}. A normatização criada estabelece:
\begin{itemize}
	\item a inserção do último sobrenome do autor em caixa alta seguido de vírgula e do prenome ou prenomes em caixa baixa, sem acrescentar anos de nascimento e morte;
	\item a sequência de inscrição: nome do premiado e autor do desenho do diploma, de acordo com a norma anterior, para a coleção de diplomas;
	\item o registro do local de assinatura na obra dos respectivos autores em espaço designado no campo.
\end{itemize}

\subsubsection{Título}
Os títulos foram extraídos de dois tipos de fontes: 
\begin{description}
	\item[fonte primária ] transcrição da própria obra;  
	\item[fonte secundária] por atribuição de especialista e de catalogador, fundamentados em pesquisa iconográfica tendo como referência bibliografia especializada.
\end{description}

Grande parte das obras do acervo do MDJVI não apresentam títulos pois compõe os conjuntos de material didático, exercícios de aulas e provas de concursos. Sendo assim a inúmeros títulos foi aplicado o critério de atribuição, adotando-se para a identificação da obra a forma mais simples e direta - \textbf{identificação do tema} ou da própria imagem, acrescidas informações complementares pertinentes, de modo a dar mais clareza ao processo de identificação.

Esta informações complementares seguem os seguintes critérios:
\begin{itemize}
	\item as cópias de obras primas famosas receberam os mesmos títulos pelos quais são conhecidos, seguidos da observação entre parênteses - (\textit{cópia de...}), exemplos:\\ 
	Vênus Anadiomene (\textit{cópia de Lisipo}), \\
	Hércules Farnese (\textit{cópia de Praxíteles}), \\
	Napoleão em Jafa (\textit{cópia de Gros}).
	\item os temas mitológicos, históricos e religiosos, claramente reconhecidos pela sua representação ou pelos atributos utilizados pelos personagens receberam os títulos convencionais, tais como: \textit{Sansão e Dalila}, \textit{Moisés}, \textit{Pigmalião e Galatéia}, \textit{A morte de Sócrates}, \textit{A crucificação de Cristo}.
	\item representações do nu apresentam no título o complemento a que se refere sua técnica de representação acadêmica: \textit{Nu feminino (academia)}, \textit{Nu feminino (academia - 2 esboços)}.
	\item em determinadas obras foi acrescido ao título atribuído notas extraídas da própria obra mas que sozinhas não determinavam seu conteúdo: \textit{Busto feminino - ``grand étude aux deux crayons n. 44''}.
	\item cópias de detalhes arquitetônicos de construção de monumentos renomados ou ainda aquelas identificadas no suporte pelo próprio autor: \textit{Medalhão da portada da igreja do Carmo (cópia de Aleijadinho, São João del Rei)}.
	\item na \textbf{Subclasse Escultura} devido a grande quantidade de cópias do período clássico foi utilizado o recurso de identificação complementar registrando-se o tipo e o período: \textit{Ângulo de capitel - alto relevo (arte gótica)}, \textit{Antefixo (arte grega)}.
	\item na \textbf{Subclasse Desenho} as cópias de esculturas foram identificadas também de forma complementar: \textit{``São Tiago'' (cópia de escultura)}, \textit{Atleta grego (cópia de escultura)}.
	\item na \textbf{Subclasse Documento diploma} o campo \textbf{Título} foi preenchido com o nome da pessoa premiada encabeçado pela nomenclatura Diploma: Diploma de João Gelabert de Simas.
\end{itemize}

Como norma ortográfica foi definido que somente os títulos e informações complementares extraídos de fonte primária entrariam entre aspas, sendo regra geral para todos os títulos a inscrição da primeira letra em maiúscula e as demais minúsculas, salvo nomes de cidades, acidentes geográficos, pessoas e correlatos, exemplos: ``\textit{Interior de um rancho}'', ``\textit{Igreja matriz da vila de Aquiras}'', Vista da matriz e do Santo Cruzeiro - Ceará, Flor, Serra do Boqueirão de Lavras, Busto feminino - ``grand étude aux deux crayons n.44''.

\subsubsection{Data}
Campo destinado ao registro da data de execução da obra. Procurou-se não deixar o item datação sem uma referência, se determinando, quando possível, década, ano e século. O processo de preenchimento deste campo foi realizado se extraindo a informação da própria obra, muitas vezes com a utilização de lentes de aumento e recursos técnicos apropriados: outras vezes a partir de pesquisa iconográfica tomando-se como base a produção do autor, pois que inúmeras obras referem-se a trabalhos realizados em épocas específicas da formação deste, destacando-se as premiações acadêmicas (Medalha de ouro, Medalha de Prata, Menção honrosa), os Prêmios de viagem ao exterior, os envios de Pensionistas, exposições oficiais, salões, concursos para magistério, encomendas oficiais; por confrontação foi atribuído o período de execução. Outra fonte utilizada para o preenchimento deste campo foram as biografias.

A inscrição dos dígitos referentes a data foi feita considerando a não identificação da década - \textbf{18\_\_}, a não identificação do ano: \textbf{195\_} e nos trabalhos onde não foi possível nenhuma forma criteriosa de identificação adotou-se a indicação \textit{s/d (sem data)} para evitar atribuições disparatadas.

\subsubsection{Local}
Campo destinado ao registro preciso do local onde a obra foi executada, sendo a informação extraída da própria obra. Como norma de inserção de dados estipulou-se o uso de maiúscula na primeira letra da informação.

\subsubsection{Técnica/material}
Para o registro dos dados deste campo tomou-se como referência o \textit{Sistema de Informação do Museu Nacional de Belas Artes} ( publicação do Patrimônio Artístico Nacional - 1995). Esta opção fundamentou-se na constatação de características comuns existentes entre os acervos do \textbf{MDJVI} e do \textbf{Museu Nacional de Belas Artes}.

Na aplicação das classificações, para os trabalhos bidimensionais, subclasses \textbf{Desenho}, \textbf{Pintura}, \textbf{Gravura}, \textbf{Documento fotográfico} e D\textbf{ocumento diploma}, especificou-se primeiro o processo e por último o suporte: \textit{sanguínea/papel}, \textit{óleo/tela}, \textit{buril/papel}, \textit{PB/papel}, \textit{semi off-set/papel}.

A classificação da \textbf{Subclasse Escultura} foi feita destacando-se os materiais em primeira ordem, uma vez que a técnica se encontra implícita: \textit{bronze}, \textit{cobre}, \textit{prata}, \textit{marfim}, \textit{madeira}, \textit{mármore}, \textit{resina de poliéster}, \textit{gesso}, \textit{terracota}, \textit{prata e madeira}, \textit{ágata e jade} etc. Como as peças em gesso constituem 80\% da coleção de esculturas, para este material foi também identificada a técnica: \textit{gesso patinado}, \textit{moldagem}, \textit{talha e relevo}.

Para as demais classes, destacou-se os materiais empregados na confecção das peças: \textit{Algodão}, \textit{Algodão acetinado}, \textit{Âmbar}, \textit{Cristal e prata}, \textit{Plumas (papo de cisne)/madrepérola}, \textit{seda e fio de prata}; também apresentando a técnica quando identificada: \textit{Bordado}, \textit{Jato de areia}, \textit{Biscuit}, \textit{Tear jacar} e \textit{Valenciana}. Estipulou-se como norma o registro da primeira letra da informação em maiúscula.

\subsubsection{Dimensões}
A medição de uma obra é realizada para cuidar de sua identificação e
segurança, para o dimensionamento do espaço e da carga exigida para sua exposição, objetivando a sua guarda, a confecção de embalagem e seu transporte.

Para o preenchimento do campo \textbf{Dimensões} foram obedecidas as normas internacionais para obras bidimensionais e tridimensionais, que estipulam a seguinte ordem de entrada dos dados: 
\textit{altura, largura, profundidade,} registradas em centímetros. Para as obras \textbf{circulares} é determinada a medida do diâmetro sendo colocado a seguir do valor a letra ``\textbf{d}'' minúscula entre parênteses - \textbf{(d)} e sua espessura; para as obras \textbf{ovais} são registrados os valores de menor e maior larguras acompanhados pela letra ``\textbf{o}'' entre parênteses \textbf{(o)} e sua espessura.

As dimensões das \textbf{obras emolduradas} e das \textbf{esculturas} com base foram registradas tomando primeiramente só o suporte ou peça e outra acrescida da moldura ou base: obra emoldurada - 55,0 x 46,0 cm - c/baguete: 57,2 x 48,0; escultura com base - 25,0 x 16,0 x 22,0cm - c/base: 35,0 x 28,8 x 17,0 cm. 

Para as \textbf{gravuras} foram registradas as dimensões do suporte e do campo impresso separadamente: 35,3 x 27,0 cm - ci: 21,0 x 14,0 cm. 

Para a entrada destes dados ainda segue-se a norma de entrada da primeira medida seguida de espaço, do sinal \textbf{x}, espaço e entrada da segunda e/ou terceira medidas. Para as peças de \textbf{gravura} destaca-se a dimensão do campo impresso identificando-o como \textbf{ci} seguido de dois pontos e da dimensão inserida com a mesma norma apresentada anteriormente, aplicando-se a mesma norma para obras com base que se identifica c/base, com moldura - c/moldura, com baguete - c/baguete.

\subsubsection{Concursos, Exposições, Premiações, Pensionistas}
Campo destinado ao registro de dados extraídos da própria obra ou coletados em bibliografia pertinente, codificados através de índice que destaca os assuntos que geraram a nomenclatura do campo. Seus dados complementam-se com os do campo \textbf{Observação} que também é destinado aos dados mais específicos que se remetem a este campo. Listagem em anexo.

\subsubsection{Tema}
Registro de termo que descreve o conteúdo temático da obra tendo como referência o \textit{título} e a \textit{imagem}. Utilizado como mais um instrumento de pesquisa, adotado somente para as classes \textbf{Artes visuais} e \textbf{Construção}. Listagem em anexo.

\subsubsection{Localização}
Campo de registro do local determinado onde a peça se encontra no espaço museográfico, alterando-se de acordo com a movimentação da mesma. Para a melhor formatação das informações foram criados códigos de localização (ver anexo).

\subsubsection{Aquisição}
Destinado a registrar o ano e o modo de aquisição da obra.

\subsubsection{Estado de conservação}
Para determinar o estado de conservação da obra foram estipuladas três categorias:
\begin{enumerate}
	\item \textbf{Bom}	- para as obras estáveis e sem grandes problemas;
	\item \textbf{Regular} - para as obras que têm danos passíveis de serem tratados;
	\item \textbf{Ruim} - para as que estão muito danificadas, inclusive, com perda de suporte ou mutiladas, às vezes totalmente fragmentadas.
\end{enumerate}

\subsubsection{Movimentação}
Para registro do histórico de movimentação da obra que eventualmente venham a ocorrer.

\subsubsection{Observações}
Reservado àquelas informações mais específicas da obra e para complementação de dados do campo \textbf{Concursos}, \textbf{Exposições}, \textbf{Premiações}, \textbf{Pensionistas}.
Registra-se neste campo:
\begin{itemize}
	\item \textbf{inscrições} que constam na obra atestando propriedade ou origem da peça no caso das moldagens de gesso, geralmente com plaquetas incrustadas com o nome do museu onde se encontra a obra original que foi reproduzida, e dedicatórias;
	\item detalhamento da forma de \textbf{aquisição} quando pertence à \textit{Coleção Ferreira das Neves}, doada à antiga Escola Nacional de Belas Artes em 1947 pelo colecionador português \textit{Jerônimo Ferreira ads Neves}, registrando-a com as iniciais maiúsculas \textbf{CFN}.
	\item \textbf{número da Exposição geral}; 
	\item \textbf{seção da Exposição geral};
	\item \textbf{número da obra na Exposição};
	\item \textbf{aula} em que o premiado estava matriculado;
	\item \textbf{informações complementares} sobre o tipo de medalha;
	\item \textbf{identificação} do trabalho do autor do desenho do diploma;
	\item \textbf{transcrição}, no caso das gravuras que possuem mais de um autor, das funções que cada autor desempenhou na elaboração da obra;
	\item \textbf{dúvidas}.
\end{itemize}


\subsubsection{Relatórios técnicos}
Como etapa do desenvolvimento do projeto, e de consequente uso do \textit{programa Access} constatou-se a necessidade de criação de relatórios técnicos parciais para consulta do acervo. Constituem-se em relatórios básicos que concentram dados relativos às atividades técnicas da área de Museologia como: a catalogação que é uma análise de maior profundidade que identifica e relaciona os bens culturais através de estudo; a conservação que consiste em tratamento preventivo à restauração; curadoria que trata da disposição do acervo para o público; e a restauração que é a intervenção técnica para reabilitação da obra. 

Para elaboração de tais relatórios foram combinados campos da ficha de registro para construir informações significativas que atendem às atividades técnicas gerais determinadas.
\begin{itemize}
	\item Autoria, Co-autoría, Título, Concursos, Exposições, Pensionistas, Observação;
	\item Conservação, Data, Técnica/materíal, Localização;
	\item Conservação, Técnica/materíal, Dimensões, Localização;
	\item Conservação, Técnica/materíal, Localização;
	\item Conservação, Técnica/materíal, Observação, Localização;
	\item Subclasse, Autoria, Co-autoría, Titulo, Data;
	\item Tema, Autoria, Co-autoría, Concursos, Observação.
\end{itemize}


\section{ANEXOS DO ACERVO MUSEOLÓGICO}
\subsection{Classes/Subclasses}
\begin{itemize}
	\item Amostras/Fragmentos
	\item Artes visuais
	\begin{itemize}
		\item Construção artística
		\item Desenho
		\item Desenho arquitetônico
		\item Escultura
		\item Gravura
		\item Pintura
	\end{itemize}
	\item Caça/Guerra 
	\begin{itemize}
		\item Acessório de armaria
		\item Arma
	\end{itemize}
    \item Comunicação
    \begin{itemize}
    	\item Documento diploma
    	\item Documento fotográfico
    \end{itemize}
	\item Construção
	\begin{itemize}
		\item Fragmento de construção
	\end{itemize}
	\item Embalagens/Recipientes
	\item Insígnia
	\begin{itemize}
		\item Bandeira
	\end{itemize}
	\item Interiores
	\begin{itemize}
		\item Acessório de interiores
		\item Objeto de iluminação
		\item Peça de mobiliário 
		\item Utensílio de cozinha/mesa
	\end{itemize}
	\item Lazer/Desporto
	\item Medição/Registro/Observação/Processamento
	\begin{itemize}
		\item Instrumento de precisão/ótico
	\end{itemize}
	\item Objetos cerimoniais
	\begin{itemize}
		\item Objeto comemorativo
		\item Objeto de culto
		\item Panegírico
	\end{itemize}
	\item Objetos pessoais
	\begin{itemize}
		\item Acessório de indumentária
		\item Artigo de tabagismo
		\item Artigo de toalete
		\item Objeto de adorno
		\item Objeto de auxílio/conforto pessoais
		\item Objeto de devoção pessoal
		\item Peça de indumentária
	\end{itemize}
	\item Trabalho
	\begin{itemize}
		\item Equipamento de artistas/artesãos
		\item Equipamento de comunicação escrita
		\item Equipamento de fiação/tecelagem
		\item Instrumento musical
	\end{itemize}
\end{itemize}

\subsection{Abreviaturas de localização de assinaturas nas obras}
 
