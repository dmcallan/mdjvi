\postextualchapter{Projeto Original}

\section{Introdução}
O \textbf{Museu D. João VI da Escola de Belas Artes da Universidade Federal do Rio de Janeiro} foi criado em 1979 com a finalidade de preservar a memória do ensino artístico oficial e de fomentar o estudo e a pesquisa da História da Arte Brasileira. Ele vem responder a necessidade da criação de um espaço institucional de preservação do patrimônio e memória do ensino de arte, reunindo a produção da Academia Imperial de Belas Artes, da Escola Nacional de Belas Artes e parte da história recente da Escola de Belas Artes.

O Museu abriga dois acervos distintos, um de obras de arte e outro de documentos, fontes primárias indispensáveis para o desenvolvimento de estudos e projetos de pesquisa em arte, quer no campo teórico quer no aplicado. Estes acervos são o resultado do patrimônio acadêmico produzido pela Escola no período compreendido, principalmente, entre 1820 e 1920.

Com o objetivo de implementar a atividade de pesquisa foi desenvolvido um projeto integrado - Museu D. João VI e Mestrado em História da Arte - EBA/UFRJ, financiado pelo CNPq, que tratou de realizar o inventário científico e sistemático destes dois acervos. 

O processo de sistematização culminou com a sua informatização, por meio de um convênio com o Núcleo de Computação Eletrônica da UFRJ, sendo o Núcleo responsável pela montagem global do sistema.

O trabalho de catalogação e sistematização foi elaborado de forma o mais técnica e criteriosa possível, sendo aplicados inúmeros instrumentos nas suas diversas etapas de identificação e de realização dos inventários. A descrição destes instrumentos, as metodologias aplicadas e os critérios de seleção e ordenação dos acervos são apresentados a seguir, se constituindo em documento técnico imprescindível, não só para o entendimento do processo, assim como para a ampliação da base do banco de dados.

\section{ACERVO MUSEOLÓGICO}
O acervo de obras do \textbf{Museu D. João VI (MDJVI)} tem uma importância singular seja para o estudo e o entendimento da história da formação artística no país, seja para a construção de uma história da arte brasileira. O acervo original da Academia Imperial de Belas Artes tem sua origem na contratação da \textbf{Missão Artística Francesa}, que chegando ao Brasil trouxe na bagagem uma coleção de 42 obras de artistas franceses e italianos, inaugurando a pinacoteca da antiga Academia. Soma-se ainda um número significativo de moldagens de gesso destinadas ao Curso de Escultura, servindo até hoje como material didático. 

Outras coleções foram sendo incorporadas à Academia e à Escola de Belas Artes, tais como as coleções de gravuras holandesas e francesas. Além destas encontra-se uma extensa coleção de medalhas produzidas pelo antigo Curso de Gravura de medalhas e pedras preciosas e, uma importante coleção de desenhos, formada pelos trabalhos de alunos e professores, compreendendo os Cursos de Desenho artístico, Modelo vivo e Desenho arquitetônico. O acervo se caracteriza por reunir um volume considerável de obras-documento, apresentadas nos concursos para professor, como aquelas referentes aos concursos para Prêmio de viagem. Destacam-se, ainda, inúmeras obras selecionadas e premiadas nas exposições oficiais da Academia, entre elas as \textit{Exposições Gerais de Belas Artes} (Salão).

O trabalho de sistematização do acervo museológico compreendeu duas etapas: a realização do inventário completo, com a identificação e fichamento das obras, e a informatização e criação do banco de dados.

Na primeira etapa foi feito um levantamento minucioso da fonte primária, que compreende o acervo completo de artes visuais - desenhos, gravuras, esculturas e pinturas - tomando como base o \textbf{Livro de Tombo do Museu D. João VI} Este processo incluiu a revisão completa das peças já inventariadas, chamando a atenção para duas coleções não museológicas registradas no \textbf{Livro de Tombo}, que são: diplomas e fotografias.

Para a etapa de informatização deste \textbf{Inventário Geral} foi elaborada uma ficha de registro de obras de acordo com as normas técnicas atuais, partindo de dados básicos e universais no que concerne à coleção de artes visuais, apresentando os seguintes campos, distribuídos em duas categorias de informação: dados gerais e dados complementares.

\subsection{Dados Gerais}
\subsubsection{Número de registro}
É o \textbf{número de registro de tombamento}. A numeração respeitou parcialmente o sistema numérico do \textbf{Livro de Tombo} encontrado, seguindo portanto, a sequência natural do número de ordem deste livro, cuja última peça fora registrada sob o número \textbf{2140}. A primeira peça sem registro a ser inventariada recebeu, consequentemente, o número 2141.Para tornar o sistema de numeração mais objetivo, optou-se por manter somente o número de ordem, abandonando o modelo tripartido do livro de tombo original, ou seja, omitindo- se os dígitos referentes ao ano de entrada da peça e à sua categoria. Para as peças compostas de diversas partes adotou-se o uso do complemento alfabético: 
\begin{itemize}
	\item registro \textbf{1463 A} - Classe Objetos pessoais, Subclasse Objetos de adorno, Relógio - e 
	\item registro \textbf{1463 B} - Classe Objetos pessoais, Subclasse Objetos de
	adorno, Estojo.
\end{itemize}

\subsubsection{Classe e Subclasse}
Os conceitos de classe e subclasse adotados foram baseados no ``\textit{Thesaurus para Acervos Museológicos}'' - Fundação Pró-Memória/1987, utilizado, na atualidade, por inúmeros museus brasileiros. De acordo com este sistema a classificação obedece a um único critério, o da funcionalidade original da peça, abolindo-se as classificações antigas, que combinavam a função com materiais e categorias. 

No caso específico do \textbf{MDJVI} a classe predominante no acervo é \textbf{Artes visuais} com suas respectivas sub-classes: 
\begin{itemize}
	\item Construção artística, 
	\item Desenho, 
	\item Desenho arquitetônico, 
	\item Escultura, 
	\item Gravura e 
	\item Pintura.
\end{itemize} 

As outras classes e sub-classes foram também empregadas, mas com menos frequência em função da própria especificidade do acervo em questão que são elas, classes: 
\begin{itemize}
	\item Amostras/fragmentos, 
	\item Caça/guerra, 
	\item Comunicação, 
	\item Construção, 
	\item Embalagem/recipientes, 
	\item Insígnia, 
	\item Interiores, 
	\item Lazer/desporto,
	\item Medição/registro/observação/processamento, 
	\item Objetos cerimoniais, 
	\item Objetos pessoais e 
	\item Trabalho.
\end{itemize} 

Como norma ortográfica para as Classe e Subclasses
estipulou-se o uso de maiúscula somente na primeira letra das nomenclaturas: \textit{Classe - Artes visuais, Subclasse - Desenho arquitetônico.}

As coleções de diploma e fotografia constituíram-se em caso especial pois foram registradas no \textbf{Livro de Tombo do Museu D. João VI}, segundo a classificação prevista no \textit{Thesaurus para acervos museológicos} que prevê o uso da classe \textbf{Comunicação} para os objetos usados para transmitir informações aos seres humanos\footnote{FERREZ Helena Dodd e BIANCHINI, Maria Helena S. \textit{Thesaurus para acervos museológicos}. p. 7.}, e a subclasse \textbf{Documento} para documentos textuais, cartográficos e iconográficos. Constituindo-se, seu uso, caso específico para museus que não possuem setor técnico para tratamento deste tipo de acervo. Baseado nesta particularidade, optou-se pela continuidade do registro estabelecendo-se adaptações para o preenchimento dos campos da ficha de registro. Como norma técnica para este campo, utilizou-se o \textit{Termo} ou \textit{Nome do objeto} junto à subclasse para evitar dúvidas quanto ao acervo consultado, criando-se as classificações: \textit{Comunicação - Documento diploma e Comunicação - Documento fotográfico}.


\subsubsection{Autoria}
Campo destinado ao registro de pessoa física ou jurídica que concebeu material e/ou intelectualmente a obra, acompanhado de cronologia. O autor foi situado cronologicamente tendo como referência os \textbf{anos de nascimento e morte}, com informações coletadas do \textit{Dicionário das Artes Plásticas no Brasil}, \textit{Dicionário Brasileiro de Artistas Plásticos} e \textit{Dictionnaire des Peintres, Sculpteurs, Dessinateurs et Graveurs}, citados em referência bibliográfica do projeto em questão. Para a normatização dos dados estipulou-se:

\begin{itemize}
	\item para entrada de dados no campo registrar-se em maiúscula o último sobrenome do autor, seguido de vírgula, do prenome ou prenomes em minúscula, e dos anos de nascimento e morte entre parênteses.
	\item para o caso de autores não identificados registrar-se no campo, não identificada;
	\item para pseudônimos registrá-los entre aspas, precedidos do autor e dos anos de nascimento e morte entre parênteses;
	\item para a coleção de diplomas registrar-se a instituição ou órgão gerador do documento em maiúsculas;
	\item para a coleção de gravuras registrar-se o autor que elaborou a matriz com o último sobrenome em maiúscula seguido de vírgula, do prenome ou prenomes em minúscula e os anos de nascimento e morte entre parênteses.
\end{itemize}

\subsubsection{Assinatura}
Destina-se ao registro do local da assinatura do autor na obra. A informação sobre a localização da assinatura na obra seguiu as normas do \textit{Manual de Catalogação de Pinturas, Esculturas, Desenhos e Gravuras}, publicado pelo \textbf{Museu Nacional de Belas Artes}. Listagem em anexo.

\subsubsection{Co-autoria}
Identifica os demais autores envolvidos na criação da obra, especificamente para as subclasses de \textbf{gravura} e \textbf{diploma}. A normatização criada estabelece:
\begin{itemize}
	\item a inserção do último sobrenome do autor em caixa alta seguido de vírgula e do prenome ou prenomes em caixa baixa, sem acrescentar anos de nascimento e morte;
	\item a sequência de inscrição: nome do premiado e autor do desenho do diploma, de acordo com a norma anterior, para a coleção de diplomas;
	\item o registro do local de assinatura na obra dos respectivos autores em espaço designado no campo.
\end{itemize}

\subsubsection{Título}
Os títulos foram extraídos de dois tipos de fontes: 
\begin{description}
	\item[fonte primária ] transcrição da própria obra;  
	\item[fonte secundária] por atribuição de especialista e de catalogador, fundamentados em pesquisa iconográfica tendo como referência bibliografia especializada.
\end{description}

Grande parte das obras do acervo do MDJVI não apresentam títulos pois compõe os conjuntos de material didático, exercícios de aulas e provas de concursos. Sendo assim a inúmeros títulos foi aplicado o critério de atribuição, adotando-se para a identificação da obra a forma mais simples e direta - \textbf{identificação do tema} ou da própria imagem, acrescidas informações complementares pertinentes, de modo a dar mais clareza ao processo de identificação.

Esta informações complementares seguem os seguintes critérios:
\begin{itemize}
	\item as cópias de obras primas famosas receberam os mesmos títulos pelos quais são conhecidos, seguidos da observação entre parênteses - (\textit{cópia de...}), exemplos:\\ 
	Vênus Anadiomene (\textit{cópia de Lisipo}), \\
	Hércules Farnese (\textit{cópia de Praxíteles}), \\
	Napoleão em Jafa (\textit{cópia de Gros}).
	\item os temas mitológicos, históricos e religiosos, claramente reconhecidos pela sua representação ou pelos atributos utilizados pelos personagens receberam os títulos convencionais, tais como: \textit{Sansão e Dalila}, \textit{Moisés}, \textit{Pigmalião e Galatéia}, \textit{A morte de Sócrates}, \textit{A crucificação de Cristo}.
	\item representações do nu apresentam no título o complemento a que se refere sua técnica de representação acadêmica: \textit{Nu feminino (academia)}, \textit{Nu feminino (academia - 2 esboços)}.
	\item em determinadas obras foi acrescido ao título atribuído notas extraídas da própria obra mas que sozinhas não determinavam seu conteúdo: \textit{Busto feminino - ``grand étude aux deux crayons n. 44''}.
	\item cópias de detalhes arquitetônicos de construção de monumentos renomados ou ainda aquelas identificadas no suporte pelo próprio autor: \textit{Medalhão da portada da igreja do Carmo (cópia de Aleijadinho, São João del Rei)}.
	\item na \textbf{Subclasse Escultura} devido a grande quantidade de cópias do período clássico foi utilizado o recurso de identificação complementar registrando-se o tipo e o período: \textit{Ângulo de capitel - alto relevo (arte gótica)}, \textit{Antefixo (arte grega)}.
	\item na \textbf{Subclasse Desenho} as cópias de esculturas foram identificadas também de forma complementar: \textit{``São Tiago'' (cópia de escultura)}, \textit{Atleta grego (cópia de escultura)}.
	\item na \textbf{Subclasse Documento diploma} o campo \textbf{Título} foi preenchido com o nome da pessoa premiada encabeçado pela nomenclatura Diploma: Diploma de João Gelabert de Simas.
\end{itemize}

Como norma ortográfica foi definido que somente os títulos e informações complementares extraídos de fonte primária entrariam entre aspas, sendo regra geral para todos os títulos a inscrição da primeira letra em maiúscula e as demais minúsculas, salvo nomes de cidades, acidentes geográficos, pessoas e correlatos, exemplos: ``\textit{Interior de um rancho}'', ``\textit{Igreja matriz da vila de Aquiras}'', Vista da matriz e do Santo Cruzeiro - Ceará, Flor, Serra do Boqueirão de Lavras, Busto feminino - ``grand étude aux deux crayons n.44''.

\subsubsection{Data}
Campo destinado ao registro da data de execução da obra. Procurou-se não deixar o item datação sem uma referência, se determinando, quando possível, década, ano e século. O processo de preenchimento deste campo foi realizado se extraindo a informação da própria obra, muitas vezes com a utilização de lentes de aumento e recursos técnicos apropriados: outras vezes a partir de pesquisa iconográfica tomando-se como base a produção do autor, pois que inúmeras obras referem-se a trabalhos realizados em épocas específicas da formação deste, destacando-se as premiações acadêmicas (Medalha de ouro, Medalha de Prata, Menção honrosa), os Prêmios de viagem ao exterior, os envios de Pensionistas, exposições oficiais, salões, concursos para magistério, encomendas oficiais; por confrontação foi atribuído o período de execução. Outra fonte utilizada para o preenchimento deste campo foram as biografias.

A inscrição dos dígitos referentes a data foi feita considerando a não identificação da década - \textbf{18\_\_}, a não identificação do ano: \textbf{195\_} e nos trabalhos onde não foi possível nenhuma forma criteriosa de identificação adotou-se a indicação \textit{s/d (sem data)} para evitar atribuições disparatadas.

\subsubsection{Local}
Campo destinado ao registro preciso do local onde a obra foi executada, sendo a informação extraída da própria obra. Como norma de inserção de dados estipulou-se o uso de maiúscula na primeira letra da informação.

\subsubsection{Técnica/material}
Para o registro dos dados deste campo tomou-se como referência o \textit{Sistema de Informação do Museu Nacional de Belas Artes} ( publicação do Patrimônio Artístico Nacional - 1995). Esta opção fundamentou-se na constatação de características comuns existentes entre os acervos do \textbf{MDJVI} e do \textbf{Museu Nacional de Belas Artes}.

Na aplicação das classificações, para os trabalhos bidimensionais, subclasses \textbf{Desenho}, \textbf{Pintura}, \textbf{Gravura}, \textbf{Documento fotográfico} e D\textbf{ocumento diploma}, especificou-se primeiro o processo e por último o suporte: \textit{sanguínea/papel}, \textit{óleo/tela}, \textit{buril/papel}, \textit{PB/papel}, \textit{semi off-set/papel}.

A classificação da \textbf{Subclasse Escultura} foi feita destacando-se os materiais em primeira ordem, uma vez que a técnica se encontra implícita: \textit{bronze}, \textit{cobre}, \textit{prata}, \textit{marfim}, \textit{madeira}, \textit{mármore}, \textit{resina de poliéster}, \textit{gesso}, \textit{terracota}, \textit{prata e madeira}, \textit{ágata e jade} etc. Como as peças em gesso constituem 80\% da coleção de esculturas, para este material foi também identificada a técnica: \textit{gesso patinado}, \textit{moldagem}, \textit{talha e relevo}.

Para as demais classes, destacou-se os materiais empregados na confecção das peças: \textit{Algodão}, \textit{Algodão acetinado}, \textit{Âmbar}, \textit{Cristal e prata}, \textit{Plumas (papo de cisne)/madrepérola}, \textit{seda e fio de prata}; também apresentando a técnica quando identificada: \textit{Bordado}, \textit{Jato de areia}, \textit{Biscuit}, \textit{Tear jacar} e \textit{Valenciana}. Estipulou-se como norma o registro da primeira letra da informação em maiúscula.

\subsubsection{Dimensões}
A medição de uma obra é realizada para cuidar de sua identificação e
segurança, para o dimensionamento do espaço e da carga exigida para sua exposição, objetivando a sua guarda, a confecção de embalagem e seu transporte.

Para o preenchimento do campo \textbf{Dimensões} foram obedecidas as normas internacionais para obras bidimensionais e tridimensionais, que estipulam a seguinte ordem de entrada dos dados: 
\textit{altura, largura, profundidade,} registradas em centímetros. Para as obras \textbf{circulares} é determinada a medida do diâmetro sendo colocado a seguir do valor a letra ``\textbf{d}'' minúscula entre parênteses - \textbf{(d)} e sua espessura; para as obras \textbf{ovais} são registrados os valores de menor e maior larguras acompanhados pela letra ``\textbf{o}'' entre parênteses \textbf{(o)} e sua espessura.

As dimensões das \textbf{obras emolduradas} e das \textbf{esculturas} com base foram registradas tomando primeiramente só o suporte ou peça e outra acrescida da moldura ou base: obra emoldurada - 55,0 x 46,0 cm - c/baguete: 57,2 x 48,0; escultura com base - 25,0 x 16,0 x 22,0cm - c/base: 35,0 x 28,8 x 17,0 cm. 

Para as \textbf{gravuras} foram registradas as dimensões do suporte e do campo impresso separadamente: 35,3 x 27,0 cm - ci: 21,0 x 14,0 cm. 

Para a entrada destes dados ainda segue-se a norma de entrada da primeira medida seguida de espaço, do sinal \textbf{x}, espaço e entrada da segunda e/ou terceira medidas. Para as peças de \textbf{gravura} destaca-se a dimensão do campo impresso identificando-o como \textbf{ci} seguido de dois pontos e da dimensão inserida com a mesma norma apresentada anteriormente, aplicando-se a mesma norma para obras com base que se identifica c/base, com moldura - c/moldura, com baguete - c/baguete.

\subsubsection{Concursos, Exposições, Premiações, Pensionistas}
Campo destinado ao registro de dados extraídos da própria obra ou coletados em bibliografia pertinente, codificados através de índice que destaca os assuntos que geraram a nomenclatura do campo. Seus dados complementam-se com os do campo \textbf{Observação} que também é destinado aos dados mais específicos que se remetem a este campo. Listagem em anexo.

\subsubsection{Tema}
Registro de termo que descreve o conteúdo temático da obra tendo como referência o \textit{título} e a \textit{imagem}. Utilizado como mais um instrumento de pesquisa, adotado somente para as classes \textbf{Artes visuais} e \textbf{Construção}. Listagem em anexo.

\subsubsection{Localização}
Campo de registro do local determinado onde a peça se encontra no espaço museográfico, alterando-se de acordo com a movimentação da mesma. Para a melhor formatação das informações foram criados códigos de localização (ver anexo).

\subsubsection{Aquisição}
Destinado a registrar o ano e o modo de aquisição da obra.

\subsubsection{Estado de conservação}
Para determinar o estado de conservação da obra foram estipuladas três categorias:
\begin{enumerate}
	\item \textbf{Bom}	- para as obras estáveis e sem grandes problemas;
	\item \textbf{Regular} - para as obras que têm danos passíveis de serem tratados;
	\item \textbf{Ruim} - para as que estão muito danificadas, inclusive, com perda de suporte ou mutiladas, às vezes totalmente fragmentadas.
\end{enumerate}

\subsubsection{Movimentação}
Para registro do histórico de movimentação da obra que eventualmente venham a ocorrer.

\subsubsection{Observações}
Reservado àquelas informações mais específicas da obra e para complementação de dados do campo \textbf{Concursos}, \textbf{Exposições}, \textbf{Premiações}, \textbf{Pensionistas}.
Registra-se neste campo:
\begin{itemize}
	\item \textbf{inscrições} que constam na obra atestando propriedade ou origem da peça no caso das moldagens de gesso, geralmente com plaquetas incrustadas com o nome do museu onde se encontra a obra original que foi reproduzida, e dedicatórias;
	\item detalhamento da forma de \textbf{aquisição} quando pertence à \textit{Coleção Ferreira das Neves}, doada à antiga Escola Nacional de Belas Artes em 1947 pelo colecionador português \textit{Jerônimo Ferreira ads Neves}, registrando-a com as iniciais maiúsculas \textbf{CFN}.
	\item \textbf{número da Exposição geral}; 
	\item \textbf{seção da Exposição geral};
	\item \textbf{número da obra na Exposição};
	\item \textbf{aula} em que o premiado estava matriculado;
	\item \textbf{informações complementares} sobre o tipo de medalha;
	\item \textbf{identificação} do trabalho do autor do desenho do diploma;
	\item \textbf{transcrição}, no caso das gravuras que possuem mais de um autor, das funções que cada autor desempenhou na elaboração da obra;
	\item \textbf{dúvidas}.
\end{itemize}


\subsubsection{Relatórios técnicos}
Como etapa do desenvolvimento do projeto, e de consequente uso do \textit{programa Access} constatou-se a necessidade de criação de relatórios técnicos parciais para consulta do acervo. Constituem-se em relatórios básicos que concentram dados relativos às atividades técnicas da área de Museologia como: a catalogação que é uma análise de maior profundidade que identifica e relaciona os bens culturais através de estudo; a conservação que consiste em tratamento preventivo à restauração; curadoria que trata da disposição do acervo para o público; e a restauração que é a intervenção técnica para reabilitação da obra. 

Para elaboração de tais relatórios foram combinados campos da ficha de registro para construir informações significativas que atendem às atividades técnicas gerais determinadas.
\begin{itemize}
	\item Autoria, Co-autoría, Título, Concursos, Exposições, Pensionistas, Observação;
	\item Conservação, Data, Técnica/material, Localização;
	\item Conservação, Técnica/material, Dimensões, Localização;
	\item Conservação, Técnica/material, Localização;
	\item Conservação, Técnica/material, Observação, Localização;
	\item Subclasse, Autoria, Co-autoria, Titulo, Data;
	\item Tema, Autoria, Co-autoria, Concursos, Observação.
\end{itemize}


\section{ANEXOS DO ACERVO MUSEOLÓGICO}
\subsection{Classes/Subclasses}
\begin{itemize}
	\item Amostras/Fragmentos
	\item Artes visuais
	\begin{itemize}
		\item Construção artística
		\item Desenho
		\item Desenho arquitetônico
		\item Escultura
		\item Gravura
		\item Pintura
	\end{itemize}
	\item Caça/Guerra 
	\begin{itemize}
		\item Acessório de armaria
		\item Arma
	\end{itemize}
    \item Comunicação
    \begin{itemize}
    	\item Documento diploma
    	\item Documento fotográfico
    \end{itemize}
	\item Construção
	\begin{itemize}
		\item Fragmento de construção
	\end{itemize}
	\item Embalagens/Recipientes
	\item Insígnia
	\begin{itemize}
		\item Bandeira
	\end{itemize}
	\item Interiores
	\begin{itemize}
		\item Acessório de interiores
		\item Objeto de iluminação
		\item Peça de mobiliário 
		\item Utensílio de cozinha/mesa
	\end{itemize}
	\item Lazer/Desporto
	\item Medição/Registro/Observação/Processamento
	\begin{itemize}
		\item Instrumento de precisão/ótico
	\end{itemize}
	\item Objetos cerimoniais
	\begin{itemize}
		\item Objeto comemorativo
		\item Objeto de culto
		\item Panegírico
	\end{itemize}
	\item Objetos pessoais
	\begin{itemize}
		\item Acessório de indumentária
		\item Artigo de tabagismo
		\item Artigo de toalete
		\item Objeto de adorno
		\item Objeto de auxílio/conforto pessoais
		\item Objeto de devoção pessoal
		\item Peça de indumentária
	\end{itemize}
	\item Trabalho
	\begin{itemize}
		\item Equipamento de artistas/artesãos
		\item Equipamento de comunicação escrita
		\item Equipamento de fiação/tecelagem
		\item Instrumento musical
	\end{itemize}
\end{itemize}

\subsection{Abreviaturas de localização de assinaturas nas obras}
  	 \begin{tabbing}
  	 	\hspace{0.7cm}\=\hspace{8cm}\=\hspace{1cm}\=\kill
  	 c	\> - centro \> vc \> - verso centro \\ 
  	 cd	\> - centro à direita \> vcd \> - verso centro direita\\ 
  	 ce	\> - centro à esquerda \> vce \> - verso centro à esquerda\\
  	 cid\> - canto inferior direito\> vcid \> - verso canto inferior direito\\ 
  	 cie\> - canto inferior esquerdo \> vcie \> - verso canto inferior esquerdo\\ 
  	 csd\> - canto superior direito \> vcsd \> - verso canto superior direito\\
  	 cse\> - canto superior esquerdo \> vcse \> - verso canto superior esquerdo\\ 
  	 ebc\> - embaixo no centro \> vebc \> - verso embaixo no centro\\ 
  	 ebd\> - embaixo à direita \> vebd \> - verso embaixo à direita\\
  	 ebe\> - embaixo à esquerda\> vebe \> - verso embaixo à esquerda\\ 
  	 ecc\> - em cima no centro \> vecc \> - verso em cima no centro\\ 
  	 ecd\> - em cima à direita \> vecd \> - verso em cima à direita\\ 
  	 ece\> - em cima à esquerda \> vece \> - verso em cima à esquerda
  	 \end{tabbing} 	
 	
\subsection{Temas}

\begin{tabbing}
	\hspace{8,7cm}\=\hspace{1cm}\=\kill
	Abstrato \> Figura Humana \\ 
	Alegórico	\> Funerário\\ 
	Anatômico \> Gênero\\
	Animalista \> Heráldico\\
	Arquitetura \> Histórico \\ 
	Caricatural	\> Literário\\ 
	Cenografia \> Marinha\\
	Composição geométrica \> Mitológico\\
	Decoração \> Natureza morta \\ 
	Documental	\> Nu\\ 
	Elemento arquitetônico \> Outros\\
	Elemento arquitetônico-fragmento (para  \> Paisagem\\   
	capitéis e talha de igreja) \> Religioso\\
	Elemento arquitetônico-estudo (para  \> Retrato\\   
	cópias de gesso) \> Social\\
	Estudo de composição  \> \\
\end{tabbing}

\subsection{Técnicas e materiais}

\begin{tabbing}
	\hspace{8,7cm}\=\hspace{1cm}\=\kill
	\textbf{A} \>  \\ 
	Aço e bronze dourado	\> Água tinta, água forte e relevo/papel\\ 
	Acrílica/eucatex \> Água tinta, buril e ponta seca/papel\\
	Acrílica/madeira \> Água tinta/papel\\
	Acrílica/tela  \> Aguada de nanquim/papel\\ 
	Ágata \> Aguada de sépia/papel \\ 
	Ágata/jade	\> Aguada, aquarela e giz/papel\\ 
	Água forte e água tinta/papel \> Algodão\\
	Água forte, água tinta e ponta seca/papel \> Algodão acetinado \\
	Água forte, verniz mole e relevo/papel \> Âmbar \\ 
	Água forte/papel	\> Aquarela e lápis de cor/papel\\ 
	Água tinta e água forte/papel \> Aquarela e pastel/papel\\
	Água tinta, água forte e ponta seca/papel  \> Aquarela/papel\\	   
\end{tabbing}

\begin{tabbing}
	\hspace{8,7cm}\=\hspace{1cm}\=\kill
	\textbf{B} \>  \\ 
	Bico de pena e talha/marfim	\> prata e lantejoula\\
	Biscuit \> Bordado/tecido, seda, ouro e prata\\
	Bordado/contas pretas \> Bordado/veludo e fio metálico\\
	Bordado/couro, pelica, seda, fio de ouro e \> Bronze\\ 
	prata \>  Bronze dourado/madeira\\ 
	Bordado/linha de seda e linho \> Bronze dourado/mármore\\ 
	Bordado/seda \> Bronze/cerâmica\\
	Bordado/seda, fio de ouro e galão \> Bronze/madeira \\
	Bordado/seda, fio de ouro e prata \> Bronze/mármore \\ 
	Bordado/seda, fio de ouro, prata e bordado	\> Bronze/pedra \\ 
	Bordado/seda, franja, fio de ouro e \> Buril e água forte/papel\\
	lantejoulas  \> Buril e ponta seca/papel\\
	Bordado/tafetá, fio de ouro e prata  \> Buril e têmpera/papel\\
	Bordado/tecido, seda, fio de ouro,  \> Buril/papel\\	   
\end{tabbing}

\begin{tabbing}
	\hspace{8,7cm}\=\hspace{1cm}\=\kill
	\textbf{C} \>  \\ 
	Cabelo trançado/metal dourado	\> Cobre\\
	Carvão \> Contas, vidro e metal\\
	Carvão e crayon/papel \> Cópia fotostática/papel\\
	Carvão e giz/papel \> Couro lavrado/madeira e metal\\ 
	Carvão e grafite/papel \>  Couro, cetim e madeira\\ 
	Carvão e sanguínea/papel \> Couro, veludo e seda\\ 
	Carvão, crayon e giz/papel \> Crayon e aquarela/papel\\
	Carvão, crayon e grafite/papel \> Crayon e carvão/papel \\
	Carvão, crayon e pastel/papel \> Crayon e giz branco/papel \\ 
	Carvão, crayon e sanguínea/papel \> Crayon e giz/papel \\ 
	Carvão, crayon e sépia/papel \> Crayon e grafite/papel\\
	Carvão, crayon, sanguínea e giz/papel  \> Crayon e pastel/papel\\
	Carvão, giz e crayon/papel  \> Crayon e sanguínea/papel\\
	Carvão, giz e grafite/papel  \> Crayon e sépia/papel\\
	Carvão, giz e sanguínea/papel \> Crayon, carvão e giz/papel\\
	Carvão, giz e sépia/papel \> Crayon, carvão e sanguínea/papel\\
	Carvão, grafite e aquarela/papel \> Crayon, carvão, sanguínea e giz/papel\\
	Carvão, pastel e giz/papel \> Crayon, giz e grafite/papel\\ 
	Carvão, sanguínea e giz/papel \>  Crayon, giz e sanguínea/papel\\ 
	Carvão/papel \> Crayon, pastel e carvão/papel\\ 
	Carvão/tela \> Crayon, sanguínea e pastel/papel\\
	Cerâmica \> Crayon/papel \\
	Chumbo e mármore \> Cristal \\ 
	Chumbo e pedras sintéticas	\> Cristal e prata (tampa) \\	   
\end{tabbing}

\begin{tabbing}
	\hspace{8,7cm}\=\hspace{1cm}\=\kill
	\textbf{E} \>  \\ 
	Ébano, marfim e prata	\> Esmalte/metal\\
	Encáustica/cartão \> Esmalte/prata\\
	Esmalte \> \\	   
\end{tabbing}

\begin{tabbing}
	\hspace{8,7cm}\=\hspace{1cm}\=\kill
	\textbf{F} \>  \\ 
	Faiança	\> Folha de Flandres\\
	Ferro \> Folha de Flandres/couro\\
	Fio de seda pura/passamanaria \> Fotolitografia/papel \\	
	Fio de seda/adamascado \> Fotopintura/tela\\   
\end{tabbing}

\begin{tabbing}
	\hspace{8,7cm}\=\hspace{1cm}\=\kill
	\textbf{G} \>  \\ 
	Gesso	\> Grafite e sanguínea/papel\\
	Gesso patinado \> Grafite, crayon e giz/papel\\
	Gesso/madeira \> Grafite, crayon e sanguínea/papel \\	
	Gesso/moldagem \> Grafite, nanquim, aquarela e carvão/papel\\
	Giz, carvão e sanguínea/papel \> Grafite, sanguínea e lápis/papel\\
	Glíptica/ouro e pedra \> Grafite/papel\\
	Glíptica/pedra \> Granito \\	
	Glíptica/pedra e madeira \> Gravura/marfim\\   
	Glíptica/prata e pedra	\> Gravura/marfim e metal dourado\\
	Grafite e aquarela/papel \> Gravura/papel e vidro\\
	Grafite e crayon/papel \> Guache c/ açúcar e relevo/papel \\	
	Grafite e giz/papel \> Guache e aquarela/papel\\ 
	Grafite e lápis/papel \> Guache/marfim, chifre, couro, metal e vidro \\	
	Grafite e nanquim/papel \> Guache/marfim, ouro e chifre\\      
\end{tabbing}

\begin{tabbing}
	\hspace{8,7cm}\=\hspace{1cm}\=\kill
	\textbf{H} \>  \\ 
	Heliogravura/papel	\> \\   
\end{tabbing}

\begin{tabbing}
	\hspace{8,7cm}\=\hspace{1cm}\=\kill
	\textbf{I} \>  \\ 
	Impressão e lápis de cor/papel \> Impressão/seda\\ 
	Impressão/papel \> Incrustado/ouro e tartaruga \\	
	Impressão/papel, tecido e madeira \> \\    
\end{tabbing}

\begin{tabbing}
	\hspace{8,7cm}\=\hspace{1cm}\=\kill
	\textbf{J} \>  \\ 
	Jato de areia/cristal	\> \\   
\end{tabbing}

\begin{tabbing}
	\hspace{8,7cm}\=\hspace{1cm}\=\kill
	\textbf{L} \>  \\ 
	Laca e folha de ouro/madeira \> Linho\\ 
	Laca/metal \> Linoleogravura/papel \\
	Lápis de cor e carvão/papel \> Litografia a vapor/papel\\ 
	Lápis de cor/tecido \> Litografia/papel \\	
	Linha enrolada com fio metálico \> \\    
\end{tabbing}

\begin{tabbing}
	\hspace{8,7cm}\=\hspace{1cm}\=\kill
	\textbf{M} \>  \\ 
	Madeira	\> Maneira negra/papel\\
	Madeira (jacarandá) \> Marfim\\
	Madeira e apliques de bronze \> Marfim e veludo \\	
	Madeira e metal \> Marfim/madeira\\
	Madeira e papier maché \> Mármore\\
	Madeira pintada \> Massa acrílica e tinta acrílica/eucatex\\
	Madeira policromada \> Metal \\	
	Madeira vazada e pintada \> Metal dourado\\   
	Madeira, bronze e espelho	\> Metal e técnica mista/papel\\
	Madeira, couro e veludo \> Metal e verniz de álcool/papel\\
	Madeira, couro e veludo roxo \> Metal prateado \\	
	Madeira, mármore e metal dourado \> Metal, água tinta e relevo/papel\\ 
	Madeira, metal e tecido \> Metal, água tinta e verniz mole/papel \\	
	Madeira, metal e veludo \> Metal/esmalte\\
	Madeira, palhinha e bronze dourado \> Metal/madeira \\	
	Madeira/bronze dourado \> Metal/papel\\
	Madeira/gesso \> Metal/vitral\\
	Madeira/metal \> Moldagem/gesso\\
	Madeira/palhinha \> Moldagem/gesso bronzeado \\	
	Madeira/prata \> Moldagem/gesso e madeira\\   
	Madeira/tecido	\> Moldagem/gesso patinado\\
	Madeira/veludo \> Mosaico/mármore amarelo\\
	Madeira/vidro \> Mosaico/mármore preto \\	
	Madrepérola \> Mosaico/metal dourado\\ 
	Madrepérola e metal \> Mosaico/metal e granito \\	
	Maneira de crayon/papel \> \\        
\end{tabbing}

\begin{tabbing}
	\hspace{8,7cm}\=\hspace{1cm}\=\kill
	\textbf{N} \>  \\ 
	Nanquim e aquarela/papel	\> Nanquim e sépia/papel\\
	Nanquim e grafite/papel \> Nanquim e tinta ferrogálica/papel\\
	Nanquim e guache/papel \> Nanquim, grafite e aquarela/papel \\	
	Nanquim e pastel/papel \> Nanquim/papel\\
\end{tabbing}

\begin{tabbing}
	\hspace{8,7cm}\=\hspace{1cm}\=\kill
	\textbf{O} \>  \\ 
	Off-set/papel	\> Óleo/papel e madeira\\
	Óleo e carvão/tela e papelão \> Óleo/tela\\
	Óleo/cartão \> Óleo/tela (marruflagem) \\	
	Óleo/cartão e tela \> Óleo/tela e cartão\\
	Óleo/cobre	\> Óleo/tela e madeira\\
	Óleo/eucatex \> Óleo/tela e papelão\\
	Óleo/madeira \> Ouro, brilhante e platina \\	
	Óleo/marfim \> Ouro, esmalte e pedras preciosas\\
	Óleo/metal \> Ouro, esmalte, vidro e diamante \\	
	Óleo/papel e cartão \> \\
\end{tabbing}

\begin{tabbing}
	\hspace{8,7cm}\=\hspace{1cm}\=\kill
	\textbf{P} \>  \\ 
	Papelão/couro cor grená	\> Pó de pedra\\
	Pastel oleoso/papel \> Ponta seca e água tinta/papel\\
	Pastel/papel \> Ponta seca e roulette/papel \\	
	PB/papel \> Ponta seca, berceau e buril/papel\\
	Pedra	\> Ponta seca/papel\\
	Pedra (basalto)	\> Porcelana\\
	Pedra, massa, fibra vegetal/madeira \> Porcelana branca\\
	Pedra-sabão \> Porcelana/metal dourado e esmalte \\	
	Pintura metálica policromada/cerâmica \> Prata\\
	Pintura/cerâmica	\> Prata e bronze dourado\\
	Pintura/madeira \> Prata e ouro\\
	Pintura/marfim \> Prata, ouro e esmalte \\	
	Pintura/porcelana \> Prata, vidro e esmalte\\
	Pintura/tecido, veludo e pérolas \> Prata/madeira \\	
	Plumas (papo de cisne)/madrepérola \> \\
\end{tabbing}

\begin{tabbing}
	\hspace{8,7cm}\=\hspace{1cm}\=\kill
	\textbf{R} \>  \\ 
	Relevo/gesso	\> Resina e ferro niquelado\\
	Relevo/gesso e madeira \> Resina/gesso\\
	Resina de poliester \>  \\	
\end{tabbing}

\begin{tabbing}
	\hspace{8,7cm}\=\hspace{1cm}\=\kill
	\textbf{S} \>  \\ 
	Sanguínea e carvão/papel	\> Sanguínea/papel\\
	Sanguínea e crayon/papel \> Seda e fio de ouro\\
	Sanguínea e giz/papel	\> Seda e fio de prata\\
	Sanguínea e sépia/papel \> Semi off-set/papel\\
	Sanguínea, carvão e giz/papel	\> Sépia e sanguínea/papel\\
	Sanguínea, crayon e giz/papel \> Sépia/papel\\
	Sanguínea, giz e carvão/papel \> Serigrafia/papel\\
\end{tabbing}

\begin{tabbing}
	\hspace{8,7cm}\=\hspace{1cm}\=\kill
	\textbf{T} \>  \\ 
	Tafetá de seda pura	\> Tecido\\
	Talha e relevo/madeira e metal \> Tecido e fio de ouro\\
	Talha e torneado/madeira	\> Tecido pintado/madeira\\
	Talha/madeira \> Tecido pintado/renda e madeira\\
	Talha/madeira dourada pintada (escaiola)	\> Tecido tingido\\
	Talha/marfim \> Tecido, seda, fio de ouro e fio de prata\\
	Talho doce/papel \> Têmpera/cartão\\
	Tapeçaria/gorgurão de seda bordado	\> Terracota\\
	Tear jacar/fio de seda adamascado \> Terracota/granito\\
	Tear jacar/fio de seda puro \> Tipografia/papel\\
\end{tabbing}

\begin{tabbing}
	\hspace{8,7cm}\=\hspace{1cm}\=\kill
	\textbf{V} \>  \\ 
	Valenciana/linha	\> fibra vegetal/madeira\\
	Vidro \> Vidro/chumbo e ferro\\
	Vidro e bronze dourado	\> Vidro/chumbo, ferro e madeira\\
	Vidro leitoso pintado \> Vidro/cristal\\
	Vidro, cerâmica, pedra, massa e \> Vinílica/tela e madeira\\
\end{tabbing}

\begin{tabbing}
	\hspace{8,7cm}\=\hspace{1cm}\=\kill
	\textbf{X} \>  \\ 
	Xilogravura/papel	\> Xilogravura/papel de arroz\\
\end{tabbing}

\begin{tabbing}
	\hspace{8,7cm}\=\hspace{1cm}\=\kill
	\textbf{Z} \>  \\ 
	Zincografia/papel	\> \\
\end{tabbing}

\section{ACERVO ARQUIVÍSTICO}
\subsection{Diagnóstico}

O acervo, inicialmente, guardado em 8 arquivos e 2 armários de aço, é composto por documentos avulsos e encadernados. Os documentos avulsos achavam-se acondicionados em pastas comuns ou suspensas, ora ordenados alfabeticamente (pasta dos professores e artistas), ora por ordem cronológica (pedidos e matrículas) ou por assunto (obras).

As pastas dos professores/artistas ocupavam dois arquivos e eram formadas por documentos originais e por xerox. Estas tinham por objetivo duplicar o original, quando duas ou mais pessoas eram mencionadas: o original era arquivado em uma das pastas e as reproduções distribuídas pelas pastas das demais pessoas citadas no original. Este procedimento provocou um aumento irreal do acervo.

Quanto a instrumentos de pesquisa, não se obteve qualquer informação sobre a sua existência, ou dos critérios adotados na organização dessa parte do acervo. As consultas eram feitas diretamente nas pastas dos artistas, ordenadas alfabeticamente pelo último sobrenome.

Os encadernados (códices), guardados em dois armários de aço, não tinham uma organização lógica. Uma lista com uma breve descrição do conteúdo, com erros de identificação, indicava a localização dos mesmos no armário.

Essa listagem era colada na parte interna das portas dos armários. Este trabalho de organização foi precedido pela tentativa de melhorar o estado de conservação das capas dos mesmos, envolvendo-as em papel fino.

\subsection{Metodologia adotada}

Inicialmente prevista para ser arranjada e descrita, a documentação começou a ser efetivamente tratada em abril de 1997, com nova proposta metodológica. Em lugar do arranjo da documentação em série e subséries, a partir do exame do acervo, optou-se por descrevê-lo peça a peça, identificando-se o acervo, resultando ao final em um inventário do conjunto documental do \textbf{Museu D. João VI}.

Esta fase foi precedida pela retirada de todas as xerox, que substituíam os originais, e dos documentos copiados de outras instituições, a fim de que o arquivo representasse exclusivamente as atividades da \textbf{Academia/Escola de Belas Artes}.

As cópias xerox foram colocadas em pastas nominais, formando um arquivo complementar que pode ser consultado pelo nome de seu titular. Os documentos de outras instituições ali existentes não foram descritos nem referenciados.

As razões para a mudança metodológica se deveram: 
\begin{itemize}
	\item À exiguidade de tempo para examinar a documentação, organizá-la em séries e subséries com a correspondente ordenação cronológica ou alfabética e posteriormente descrevê-la (em abril ainda não se tinha a confirmação da renovação do projeto, por isso tínhamos que trabalhar com a possibilidade de sua interrupção, sem, no entanto, inviabilizar a organização encontrada), 
	\item À falta de condições físicas e materiais: espaço físico para 9 bolsistas examinarem, separarem e guardarem os documentos até a fase final com a descrição desses conjuntos (naquele momento ainda não haviam sido compradas as estantes e as caixas-box); 
	\item O tipo de demanda do usuário do arquivo: consultas em torno de um nome, o que já tinha direcionado a organização anterior da documentação em pastas individuais; 
	\item A possibilidade da informatização das informações, possibilitando a criação de inúmeros instrumentos de trabalho e de pesquisa, tanto para uso externo (pesquisadores) quanto interno (controle da localização do acervo, ou de sua conservação).
\end{itemize}

\subsubsection{Planilha: descrição dos campos e critérios de preenchimento}

Com a finalidade de se obter a uniformização no levantamento dos dados, adotou-se uma planilha com 13 campos a serem preenchidos:

\begin{enumerate}
	\item \textbf{Notação:}
	
	É a referência do documento, pela qual será o mesmo localizado.
	\item \textbf{Fundo:}
	
	Indica o órgão produtor/acumulador da documentação, identificado pela sua última denominação - Escola de Belas Artes.
	\item \textbf{Série:}
	
	Campo não utilizado.
	\item \textbf{Subsérie:}
	
	Campo não utilizado.
	\item \textbf{Conteúdo:}
	
	Descreve um documento ou conjunto de documentos sobre um mesmo assunto (dossiê). No caso de registro de minutas de correspondência, em alguns casos, fez-se um resumo dessas minutas em uma só planilha.
	\item \textbf{Avulso e encadernado:}
	
	Destaca o aspecto final do documento: livro/códice é o encadernado; documentos soltos ou sob a forma de processo, tratado como uma unanimidade, é o documento avulso.
	\item \textbf{Número de documentos:}
	
	É o total de documentos descritos sob uma notação ou referência. Os códices ou encadernados foram considerados como uma unidade, pois não se fez a descrição de cada um dos documentos que o compõe.
	\item \textbf{Datas-limites:}
	
	Compreende a data inicial e final do dossiê descrito sob uma notação ou, no caso de um documento, aparecerá a sua data, caso de ela existir. O campo aparecerá em branco no caso de documento sem data.
	\item \textbf{Numeração original:}
	
	Campo destinado a anotar, quando existente, o número de época de um encadernado. Tem po objetivo fornecer a sequência numérica dos livros existentes para determinados registros, como por exemplo, o de matrículas ou visualizar as falhas existentes (livros desaparecidos).
	\item \textbf{Apresentação:}
	
	Indica a forma gráfica do documento ou do dossiê, com a possibilidade de haver as 3 opções simultâneas: manuscrito, impresso ou datilografado.
	\item \textbf{Documentos especiais:}
	
	São aqueles não textuais (exceção aos recortes de jornais).
	\item \textbf{Estado de conservação:}
	
	Tem a finalidade de indicar as prioridades na restauração do acervo, bem como restringir o acesso, em razão do seu estado de conservação.
	\item \textbf{Índice onomástico:}
	
	Recupera o nome de pessoas, empresas, e repartições citadas no campo conteúdo.
	\item \textbf{Índice temático:}
	
	Recupera assuntos tratados no campo conteúdo, alguns dos quais agrupados sob um descritor.
	\item \textbf{Observação:}
	
	Reservado àquelas informações significativas que fogem à sistematização dos demais campos.
\end{enumerate}

\subsubsection{Critérios adotados no preenchimento dos campos}

\begin{itemize}
	\item \textbf{Notação:}
	
	Numeração sequencial, de 1 a 6113, atribuída a cada documento ou dossiê identificado e descrito. Esta numeração não reflete a ordenação cronológica dos mesmos, mas sim a ordem em que foram identificados e classificados. A ordem cronológica será dada com a informatização desses dados.
	
	\item \textbf{Fundo:}
	
	Recebeu a denominação do último produtor/acumulador: Escola de Belas Artes.
	
	\item \textbf{Série e Subsérie:}
	
	Campos não preenchidos por não ter sido feito o arranjo da documentação após a sua identificação.
	
	\item \textbf{Conteúdo:}
	
	Procurou-se unformizar a descrição, esquematizando-a em subcampos:
	
	\begin{itemize}
		\item \textbf{Espécie / Tipo de Documento:}
		
		Ofícios, avisos, cartas, requerimentos, assentamentos, portarias, etc. antecedida pela informação de tratar-se ou não do original (neste caso, minuta ou cópia).
		
		Em virtude da inexistência de fontes de referência específicas para a identificação dos remetentes da correspondência oficial, em especial, a dos ministros aos quais a Academia esteve subordinada, descreveu-se, algumas vezes, esse tipo de correspondência como ofício e não como aviso, espécie documental pela qual um ministro se comunica com seus subordinados.
		
		Nos casos em que o aspecto formal do documento não oferecia dúvidas quanto a sua classificação ou se reconhecia a assinatura do ministro, essa espécie documental foi descrita na sua verdadeira acepção, isto é, como aviso.
		
		Para a identificação dos nomes dos ministros que ocuparam esses cargos no período de 1808 a 1889, pode ser consultado o livro de autoria de Miguel Arcângelo Galvão, "Relação dos cidadãos que tomaram parte no governo do Brasil no período de 1808 a 15 de novembro de 1889", editado pelo Arquivo Nacional, e para o período de 1808 a 1889, "Organização e Programas Ministeriais - Regime Parlamentar no Império", pelo Barão de Javari, $3^a$ edição, Ministério da Educação e Cultura/Instituto Nacional do Livro, Brasília, 1979.
		
		A seguir mencionou-se o emissor/remetente encarregado pela expedição do documento - Seção ou Diretoria, por exemplo: aos quais estavam subordinados os serviços. Estas informações geralmente vinham impressas no documento. Ex.: Ofício do diretor da $2^a$ Diretoria do Ministério do Império para o diretor da Academia...
		
		\item \textbf{Emissor / Remetente:}
		\begin{itemize}
			\item No caso de correspondência oficial entre a Academia e o ministério ao qual estava subordinada, não se mencionou o nome do remetente e do destinatário, apenas os cargos por eles empenhados.
			
			\textbf{Ex.:} Minuta de ofício do diretor da Academia de Belas Artes para o diretor da $2^a$ Diretoria do Ministério do Império.
		
			\item No caso das minutas da correspondência expedidas pela Academia/Escola, geralmente rascunhadas em folhas soltas e com assuntos apenas esboçados, adotou-se descrevê-las resumidamente e em um único documento.
			
			\textbf{Ex.:} Minutas de ofícios do diretor da Academia sobre: nomeação de professores, concursos, vencimentos.
			
			\item Distinguem-se as minutas das cópias pelas rasuras e, geralmente, a ausência de assinatura nas primeiras.
		\end{itemize}
		
		\item \textbf{Assunto:}
		
		Descrito de forma resumida, abordando-se as principais ideias e informações ali contidas, em especial aquelas baseadas em atos legislativos ali mencionados e recuperados no índice temático em legislação.
		
		A fim de contextualizar a informação com a sua fundamentação (legislação) fez-se o levantamento desses atos legislativos, com uma breve descrição de seu conteúdo e a correspondente referência da fonte utilizada: documentos originais, registros nas repartições de origem, publicação na Coleção de Leis do Brasil ou em jornais da época, incumbidos da publicação dos atos do governo. Foram eles:
		
		\begin{tabbing}
			\hspace{0.7cm}\=\hspace{5.5cm}\=\hspace{1cm}\=\kill
			Gazeta do Rio de Janeiro	\>  \> - \> 10/09/1808 a 29/12/1821\\
			Gazeta do Rio \>\>- \> 01/01/1822 a 31/12/1822\\
			Diário do Governo	\>\>- \> 02/01/1823 a 20/05/1824\\
			Diário Fluminense \>\>- \> 21/05/1824 a 23/04/1831\\
			Diário do Governo	\>\>- \> 25/04/1824 a 28/06/1833\\
			Correio Oficial \>\>- \> 01/07/1833 a 04/08/1841\\
			Jornal do Comércio \>\>- \> 15/08/1841 a 31/08/1846\\
			Correio Mercantil \>\>- \> 01/08/1848 a 23/10/1848\\
			Diário do Rio de Janeiro	\>\>- \> 24/10/1848 a 31/12/1854\\
			Jornal do Comércio \>\>- \> 01/01/1855 a 30/09/1862\\
			Diário Oficial \>\>- \> 01/10/1862 a ...\\
		\end{tabbing}
		(LIMA, Raul - A criação do Diário Oficial. Rio de janeiro, Imp, 1979. 70 p.)
		
		Alguns documentos foram agrupados pela espécie documental, como por exemplo, os atestados: médico, de vacinação, de escolaridade, de nascimento, de idade, de naturalidade, de batismo, de nacionalidade, as certidões de nascimento, os assentamentos de batismo; outros pelo assunto, como as matrículas, ordenadas pelo último sobrenome do aluno.
		
		Dentro de cada espécie documental foram os mesmos agrupados em ordem alfabética pelo último sobrenome, à exceção dos assentamentos de batismo, pois deles não constava o sobrenome da criança, apenas o do pai e da mãe. Adotou-se então a ordem cronológica do ano de batismo, critério que também foi usado para as certidões de nascimento.
		
		Dentro de cada espécie documental, os casos de sobrenomes não identificados foram agrupados pelo prenome.
		
		\item \textbf{Anexos:}
		
		Após a descrição do conteúdo do documento ou do dossiê, assinalou-se a existência de anexos ali colocados à época de sua elaboração ou ali reunidos, por tratar-se do mesmo assunto, fazendo-se um breve resumo do seu conteúdo.
	\end{itemize}
\end{itemize}

\subsubsection{Índice Onomástico}

Destinado a recuperar nomes de pessoas, instituições e empresas constantes do campo conteúdo.

Para consultá-lo cabem duas observações:

\begin{itemize}
	\item Não foram feitas pesquisas para sanar dúvidas quanto à grafia dos nomes e alterações nas denominações de algumas instituições. Os critérios foram adotados, optando-se por uma das formas utilizadas.
	\item Foram adotados critérios visando a uniformização das entradas dos nomes no onomástico. As exceções, quando existentes, foram assinaladas logo após a regra.
\end{itemize}
\begin{enumerate}[label=(\alph*)]
	\item Adotou0se como regra geral para o nome de pessoas, a entrada pelo último sobrenome.
	
	Ex.: AMOEDO, Rodolfo
	
	No entanto, devido à diversidade de grafias de alguns nomes, ora completos, ora incompletos, para grupá-los em uma única entrada, adotou-se nestes casos, o critério de se sinalizar entre colchetes no campo conteúdo os diversos sobrenomes passíveis de serem pesquisados, quando não forem localizados no último sobrenome. Ex.:
	\begin{itemize}
		\item CONSTANT, Benjamin \textbf{(forma adotada)}
		\item MAGALHÃES, Benjamin Constant Botelho de
		\item RIOS FILHO, Adolfo Morales de los \textbf{(forma adotada)}
		\item CUADRA, Adolfo Morales de los Rios
		\item ENEIAS, Dinorá Carolina de Azevedo de Simas \textbf{(forma adotada)}
		\item SIMAS, Dinorá de Azevedo de
	\end{itemize}
	
	\item Nomes e sobrenomes de origem portuguesa foram modernizados, segundo regras estabelecidas na obra "A construção do livro", de Emanuel Araújo (Rio de Janeiro: Nova Fronteira; Brasília: INL - Instituto Nacional do Livro, 1986). Assim:
	\begin{itemize}
		\item Melo \textbf{e não} Mello
		\item Manuel \textbf{e não} Manoel
		\item Novais \textbf{e não} Novaes
		\item Sousa \textbf{e não} Souza
		\item Cardoso \textbf{e não} Cardozo
		\item Correia \textbf{e não} Correa
	\end{itemize}
	
	\item Nos sobrenomes antecedidos pela partícula DE (maiúscula), a entrada foi dada por ela e não pelo último sobrenome. Ex.:
	\begin{itemize}
		\item DE ANGELIS, Ferdinando
		\item DE MARTINO, Edoardo
	\end{itemize}

	\item Os nomes de empresas e repartições entraram na ordem direta. Ex.:
	\begin{itemize}
		\item Diário de Notícias
		\item J. L. Garnier \& CIA
		\item Conservatório de Música
	\end{itemize}

	\item Títulos de nobreza entraram pela denominação, seguido do grau. Ex.:
	\begin{itemize}
		\item MONTE ALEGRE, visconde de
	\end{itemize}
	
	\item Os nomes das instituições e dos órgãos públicos foram grafados segundo as variações ocorridas em sua denominação. Ex.:
	\begin{itemize}
		\item Conservatório de Música e Instituto Nacional de Música
		\item Inspeção Geral das Obras Públicas e Inspetoria Geral das Obras Públicas
		\item Escola de Medicina do Rio de Janeiro e Faculdade de Medicina do Rio de Janeiro
		
		Constitui exceção o Arquivo Nacional, mencionado como Arquivo Público do Império e Arquivo Público Nacional. Optou-se por adotar Arquivo Público, forma que aglutina no nome elementos tanto do período imperial quanto da República.
	\end{itemize}

	\item Quando a localização geográfica dessas instituições e/ou órgãos públicos não implicava em modificar a informação, procurou-se simplificar sua denominação. É o caso de Alfândega da Corte e Alfândega do Rio de Janeiro. Adotou-se, apenas, Alfândega.
	
	\item No caso de instituições estrangeiras com diversidade de denominações e variações quanto ao idioma, adotou-se uma das denominações e o idioma do país em que está localizada. Ex.:
	\begin{itemize}
		\item École [spéciale] des Beaux-Arts
		\item École Impériale et Spéciale des Beaux-Arts
		\item École Nationale des Beaux-Arts
		\item École Nationale et Spéciale des Beaux-Arts
		\item Escola de Belas Artes de Paris
		\item Escola Especial de Belas Artes de Paris
		\item École des Beaux-Arts de paris \textbf{(formada adotada)}
		
		Foi adotado: \textbf{École de Beaux-Arts de paris}.
	\end{itemize}

	\item Os nomes das aulas/cadeiras/disciplinas fazem parte do índice temático, inclusive as do Conservatório de Música, enquanto integrante da Academia, como uma de suas seções.
	
	\item Pintura, escultura, gravura e desenho são descritores que englobam o conceito arte/processo/técnica ou disciplina/cadeira/aula.
\end{enumerate}

\subsubsection{Índice Temático}

Destinado a recuperar os assuntos do campo conteúdo através de descritores. Descritores utilizados com a explicação da aplicação de alguns casos:

\begin{tabbing}{|p}
	\hspace{0.7cm}\=\hspace{4.7cm}\=\hspace{3cm}\=\kill
	\textbf{Acervo bibliográfico}	\>  \> usado para \> aquisição de livros e revistas.\\
	\textbf{Antecedentes sociais} \>\>usado para \> atestados de antecedentes, de idoneidade\\
		\>\> \> moral, folha corrida.\\
	\textbf{Arrecadação}	\>\>usado para \> cobranças de entradas em exposição e de\\
		\>\> \> quadros no país.\\
	\textbf{Assistência social} \>\>usado para \> pedidos de auxílio pecuniário a viúvas de\\
		\>\> \> artistas, professores e a alunos.\\
	\textbf{Bibliografia usada}	\>\>usado para \> pedidos de informações sobre livros\\
		\>\> \> adotados na Academia.\\
	\textbf{Calendário escolar} \>\>usado para \> horários, abertura e fechamento do ano\\
		\>\> \> escolar.\\
	\textbf{Cessão de instalações} \>\>usado para \> pedidos de empréstimos de salas e outras\\
		\>\> \> dependências da Academia.\\
	\textbf{Conservação de acervo} \>\>usado para \> conservação e restauração do acervo.\\
	\textbf{Conservação predial}	\>\>usado para \> medidas ligadas à limpeza do prédio.\\
	\textbf{Contratação de serviços} \>\>usado para \> contratos, concorrências, propostas para\\
		\>\> \> execução de serviços.\\
	\textbf{Corpo discente} \>\>usado para \> nomes de alunos mencionados nos\\
		\>\> \> documentos e não indexados, cada um de\\
		\>\> \> por si.\\
	\textbf{Críticas e denúncias} \>\>usado para \> delações, denúncias, reclamações ligadas à\\
	\>\> \> Academia ou a seus funcionários.\\
	\textbf{Datas especiais} \>\>usado para \> festas, comemorações, casamentos, etc.\\
	\textbf{Despesas} \>\>usado para \> gastos efetuados, recibos, pagamento de\\
	\>\> \> contas e de transporte.\\
	\textbf{Devolução de acervo} \>\>usado para \> intercâmbio de acervos.\\
	\textbf{Documentos de saúde} \>\>usado para \> atestados de vacinações e saúde.\\
	\textbf{Documentos probatórios civis} \>\>usado para \> agrupar referentes a certidões,\\
	\>\> \> assentamentos, declarações, ligados a\\
	\>\> \> registros civis (nascimento, casamento, óbito,\\
	\>\> \> idade, naturalidade).\\
	\textbf{Eleições} \>\>usado para \> escrutínio, municipal, estadual, federal, como\\
	\>\> \> também de professores honorários.\\
	\textbf{Embarcações} \>\>usado para \> indicar referências ao transporte de obras de\\
	\>\> \> arte, especialmente, as enviadas pelos\\
	\>\> \> pensionistas.\\
	\textbf{Ensino de música} \>\>usado para \> ensino genérico da música ou de um\\
	\>\> \> instrumento.\\
	\textbf{Estatutos} \>\>usado para \> regulamentos e suas reformas.\\
	\textbf{Estudos} \>\>usado para \> indicar trabalhos ou textos não identificados.\\
	\textbf{Exposições} \>\>usado para \> indicar não só exposições em sentido\\
	\>\> \> genérico, como também programas e\\
	\>\> \> catálogos das mesmas.\\
	\textbf{Frequência} \>\>usado para \> frequência de alunos, professores e\\
	\>\> \> funcionários ou atestados a ela referentes.\\
	\textbf{Habilitação profissional} \>\>usado para \> diplomas ou pedidos de registros\\
	\>\> \> profissionais e aos pedidos de\\
	\>\> \> certidões/atestados a eles referentes.\\
	\textbf{Histórico escolar} \>\>usado para \> indicar referências ao aproveitamento\\
	\>\> \> escolar, como notas, aprovações etc.\\
	\textbf{Imóveis} \>\>usado para \> referências a uso, transferência,\\
	\>\> \> reintegração, compra e venda, aluguel e\\
	\>\> \> transferência de sede.\\
	\textbf{Impostos} \>\>usado para \> pedidos de isenção de impostos na entrada\\
	\>\> \> das obras de arte enviadas pelos\\
	\>\> \> pensionistas.\\
	\textbf{Informações institucionais} \>\>usado para \> pedidos de informações ou dados sobre a\\
	\>\> \> Academia/Escola.\\
	\textbf{Júri} \>\>usado para \> as requisições de funcionários para servirem\\
	\>\> \> no Tribunal do Júri.\\
	\textbf{Legislação} \>\>usado para \> indicar menção a ato legislativo.\\
	\textbf{Materiais permanente e de} \>\>usado para \> menção a materiais para uso da Academia\\
	\textbf{consumo}\>\> \> em suas atividades administrativas (mesas,\\
	\>\> \> cadeiras, papel, etc.).\\
	\textbf{Materiais didáticos} \>\>usado para \> os materiais específicos de cada aula.\\
	\textbf{Normas e instruções} \>\>usado para \> informações destinadas a concurso,\\
	\>\> \> matrículas, questões disciplinares, etc.\\
	\textbf{Obras e edificações} \>\>usado para \> construção, ampliação, reformas, instalações\\
	\>\> \> hidráulicas e elétricas.\\
	\textbf{Pensionistas} \>\>usado para \> referências aos premiados com Prêmio de\\
	\>\> \> $1^a$ Ordem.\\
	\textbf{Pensões} \>\>usado para \> pagamento a dependentes de empregados e\\
	\>\> \> funcionários.\\
	\textbf{Prêmio de $1^a$ ordem} \>\>usado para \> referir-se à premiação dos futuros\\
	\>\> \> pensionistas\\
	\textbf{Prêmio de viagem} \>\>usado para \> premiação de viagem distinta daquela\\
	\>\> \> concedida aos pensionistas\\
	\textbf{Quadro funcional} \>\>usado para \> indicar menção a nomes de funcionários e\\
	\>\> \> professores que, na maioria das vezes, não\\
	\>\> \> entraram no índice onomástico.\\
	\textbf{Segurança patrimonial} \>\>usado para \> as questões de envolvimento de alunos e\\
	\>\> \> funcionários na prestação de serviços\\
	\>\> \> militares, especialmente na Guarda Nacional.\\
	\textbf{Sinistros} \>\>usado para \> referências à alagamentos, incêndios e\\
	\>\> \> temporais.\\
	\textbf{Urbanização} \>\>usado para \> ações de melhoramento na cidade, abertura\\
	\>\> \> de ruas, etc.\\
	\textbf{Vencimentos} \>\>usado para \> pagamentos a empregados, funcionários e\\
	\>\> \> modelos vivos.\\
	\textbf{Vestimentas} \>\>usado para \> referenciar figurino das vestes dos lentes,\\
	\>\> \> diretores e funcionários da Faculdade de\\
	\>\> \> Medicina, o uso dessas vestes professorais\\
	\>\> \> ou qualquer outra referência a trajes.\\
	\textbf{Visitas} \>\>usado para \> qualquer tipo de visitas ao país, à cidade e,\\
	\>\> \> principalmente, à Escola.\\
\end{tabbing}

\subsubsection{Instrumentos de pesquisa elaborados:}

\begin{itemize}
	\item \textbf{Inventário da documentação:} descrição cronológica da documentação existente no arquivo do Museu D. João VI, período de 1824 a 1939 (dezembro).
	
	\item \textbf{Índices onomástico e temático:} relação alfabética dos nomes e assuntos tratados no inventário.
	
	\item \textbf{Índice cronológico da legislação:} resumo do ato legislativo citado no inventário (campo conteúdo), visando a melhor compreensão do assunto ali tratado. Esses atos foram levantados a partir do descritor LEGISLAÇÃO do índice temático.
	
	\item \textbf{Índice topográfico:} tem por objetivo a localização dos documentos no depósito do arquivo do Museu D. João VI.
\end{itemize}

\subsubsection{Guarda do acervo}

A documentação acha-se armazenada em um depósito, em dependência contígua à reserva técnica do MUSEU, distribuída em oito estantes de aço, com tratamento anti-ferrugem.

Ocupa o acervo 4 conjuntos de estantes e 8 módulos, num total de 9,5 metros lineares de documentos avulsos identificados e descritos, e 3 lineares referentes a 107 códices. As caixas-box com os documentos avulsos ocupam as estantes 1 e 2 com 4 módulos e 1 prateleira; os 107 códices, as estantes 2 e 3 com 3 módulos. Dada a altura das estantes e o tamanho de alguns códices, optou-se por colocá-los na horizontal nas estantes. 

Seguem-se os códices posteriores a dezembro de 1939, que não foram objeto de classificação, na estante 4, com 1 módulo e 1 prateleira. Para consulta a este último conjunto documental, elaborou-se uma relação com a descrição do conteúdo de cada códice, período (data) e sua localização no depósito.

Os documentos avulsos posteriores a dezembro de 1939 foram mantidos em suas pastas, quando ali já estivessem. Os demais encontrados no arquivo sem nenhum tratamento foram empacotados e colocados em 2 arquivos, num total de 8 gavetas. Estes documentos não foram objeto de identificação e descrição por ultrapassarem a data-limite estabelecida para tratamento do acervo.

Os documentos em xerox, que compunham as antigas pastas dos professores/artistas, foram retiradas e guardadas em pastas nominais em um arquivo de 4 gavetas formando um arquivo complementar, que poderá ser consultado procurando-se o nome do seu titular.

\subsubsection{Acondicionamento do acervo}

Para os documentos avulsos foram confeccionados envelopes especiais, onde foram arquivadas cada unidade documental descrita. Do lado de fora, a lápis, colocou-se notação correspondente. A seguir, os envelopes foram acondicionados em caixas-box, segundo modelo que melhor se adequava às condições de armazenamento, isto é, com respiradores na parte da frente e de trás da caixa. 

Em cada caixa coube número variável de unidades de arquivamento, em virtude do conteúdo também variável, de documentos descritos em cada um desses envelopes. Através do índice topográfico é possível se calcular o número de itens documentais (envelopes) descritos em cada caixa. Da identificação de cada uma, consta seu número (ordenação sequencial de 1 a -) e os números limites das unidades de arquivamento ali guardadas.

Os códices (encadernados), após sua identificação e descrição, foram encapados em papel kraft e amarrados com cadarço. No seu interior, na primeira folha após a capa, ou em outro local, quando neste não era possível, foi colocada, a lápis, a sua notação, para preservá-la, no caso de perda da etiqueta identificadora externa do mesmo e sua consequente localização no depósito.

A documentação posterior a dezembro de 1939 não foi acondicionada em papel kraft. Manteve-se aquele já encontrado (papel fico branco), colocando-se, no entanto, uma etiqueta identificadora da sequência numérica dos mesmos e sua localização na estante, conforme listagem previamente elaborada, visto que a referida documentação tem sido, eventualmente consultada pela Secretaria da Escola de Belas Artes.

\section{ANEXOS DO ACERVO ARQUIVÍSTICO}

\subsection{Índices temático}

\begin{tabbing}
	\hspace{8,7cm}\=\hspace{1cm}\=\kill
	"$1^a$ emancipação municipal"	\> "XX"\\
	 $1^o$ Centenário da Independência do Brasil \> $22^a$ Exposição Geral de Belas Artes\\
	 $2^o$ Congresso Pan Americano Científico\> $3^a$ Exposição Geral de Belas Artes \\	
	 $3^a$ Exposição Nacional\> \\
	
	
\end{tabbing}

\begin{tabbing}
	\hspace{8,7cm}\=\hspace{1cm}\=\kill
	\textbf{A} \>  \\ 
	"A arte polonesa de 1800 até hoje"	\> "A caridade"\\
	"A dispersão dos Apóstolos" \> "A divina comédia"\\
	"A embriaguês de Noé" \> "A esquadra inglesa bordejando nos mares\\
	"A Independência do Brasil" \> de Stromboli"\\ 
	"A lagoa de Rodrigo de Freitas" \>  "A louca"\\ 
	"A morte de Catão de Utica" \> "A morte de Felipe $2^o$ da Espanha"\\ 
	"A morte de Sócrates" \> "A morte de Virgínia"\\
	"A narração de Filetas" \> "A primeira missa no Brasil" \\
	"A religião faz a felicidade dos povos" \> "A sagração de Sua Majestade o Imperador \\ 
	"A saída da vida pecaminosa"	\> em 1841" \\ 
	"A tagarela" \> "A tarde"\\
	"A Virgem com o menino Jesus"  \> Abastecimento de água\\
	Acervo bibliográfico  \> Acumulação de cargos\\
	Admissões e nomeações  \> "Adoração dos Magos"\\
	"Agamenon de volta a seu palácio..." \> Álgebra \\ 
	Alojamento de alunos e empregados	\> "Analisi dal vero per I'insegnamento del \\ 
	Anatomia \> ornato"\\
	Anatomia artística  \> Anatomia artística e fisiologia das paixões\\
	Anatomia e fisiologia  \> Anatomia e fisiologia artísticas\\
	Anatomia e fisiologia das paixões  \> "Anatomie du Gladiateurs"\\
	"Andrômeda" \> Antecedentes sociais \\ 
	"Antíope"	\> "Anuário de estatística do município Neutro" \\ 
	"Anuário dos contemporâneos Quem sou \> "Apolo de Belvedere"\\
	eu?"  \> Aposentadorias\\
	Aquisição de acervo  \> "Ária de Milão"\\
	Aritmética  \> "Armas Imperiais"\\
	Arqueologia \> Arqueologia e etnografia \\ 
	Arquiteto-desenhador	\> "Arquitetura" \\ 
	Arquitetura \> Arquitetura civil\\
	Arquitetura da arte  \> Arquitetura decorativa\\
	Arrecadação  \> "Arrufos"\\
	Arte decorativa  \> Artes aplicadas\\
	Artes e ciências \> "As duas brasileiras" \\ 
	"As tardes de Napoleão I"	\> "Assinatura da paz" \\ 
	Assistência social \> Atas\\
	"Atelier"  \> "Atlas e relatório concernente a exploração\\
	"Atlas e relatório sobre a Exposição  \> do rio São Francisco"\\
	Internacional de 1862"  \> Aula de Comércio\\
	"Aurora"  \> Avaliação de obra de arte\\		   	   
\end{tabbing}

\begin{tabbing}
	\hspace{8,7cm}\=\hspace{1cm}\=\kill
	\textbf{B} \>  \\ 
	"Bacantes em festa"	\> "Baía de Guanabara"\\
	"Baile à fantasia" \> Bandeiras\\
	"Batalha do Avaí" \> "Batalha do Campo Grande"\\
	"Batismo de Jesus Cristo" \> "Belisário esmolando..."\\ 
	"BHAC" \>  Bibliografia adotada\\ 
	"Bom tempo" \> "Busto de Antônio Carlos"\\ 
	"Busto de Benjamin Constant" \> "Busto de D. Pedro II"\\
	"Busto de Martim Francisco" \> "Busto de Valentim da Fonseca" \\		   	   
\end{tabbing}

\begin{tabbing}
	\hspace{8,7cm}\=\hspace{1cm}\=\kill
	\textbf{C} \>  \\ 
	"Cabeça de carroceiro"	\> "Cabeça de Ciociaro"\\
	"Cabeça de mulher" \> "Cabeça de São Jerônimo"\\
	"Cabral dá ao país o nome de Santa Cruz" \> "Caim odiando seu irmão Abel..."\\
	Cálculo \> Cálculo e mecânica\\ 
	Cálculo, mecânica e construção \>  Calendário escolar\\ 
	Canto \> "Carta arquitetural da cidade do Rio de\\ 
	"Carta do teatro da guerra do Paraguai" \> Janeiro"\\
	"Carta geral do Império" \> "Casamento místico de Santa Catarina" \\	
	"Cascatinha da Tijuca"	\> "Catálogo da XXIII Exposição Geral de\\
	"Catálogo do VI Salão Nacional" \> Belas Artes"\\
	"Catálogo Ilustrado del Salón Nacional de \> "Catálogo Ilustrado"\\
	Belas Artes" \> Centenário da Escola de Belas Artes\\ 
	Cessão de instalações \>  "Chefs d'oeuvre de l'Opéra français"\\ 
	Ciências acessórias \> Ciências físicas e naturais\\ 
	"Clarão de luz" \> Clarineta\\
	"Coleção dos melhores ornatos antigos em \> "Colombo"\\		   	   
	Veneza"	\> "Combate de Itapiru"\\
	"Combate naval de Riachuelo" \> Comissão de Arrolamento da População do\\
	Comissão didática \> Município da Corte\\
	Comissão Diretora da Manifestação ao \> Comissão disciplinar\\ 
	Barão do Rio Branco \> Comissão Nacional de Bela Artes\\ 
	Comissões \> Composição de arquitetura\\ 
	Composição de arquitetura, seu desenho e \> Concertos\\
	orçamento \> Concursos \\		   	   
	Condecorações	\> Congregação de professores\\
	Congresso Juvenil Artístico \> Congresso Nacional de Engenharia e\\
	Conselho docente \> Indústria\\
	Conselho escolar \> Conselho Superior de Belas Artes\\ 
	Conselho técnico-administrativo da ENBA \>  Conselho universitário\\ 
	Conservação de acervo \> Conservação predial\\ 
	Constituições do Brasil \> Contrabaixo\\
	Contratação de serviços \> "Coroação de D. Pedro II" \\	
	"Corografia Histórica do Brasil" \> Corpo discente\\ 
	 "Cristo" \> "Cristo e a adúltera"\\
	Críticas e denúncias \> Cursos \\		   	   	   	   
\end{tabbing}

\begin{tabbing}
	\hspace{8,7cm}\=\hspace{1cm}\=\kill
	\textbf{D} \>  \\ 
	"D. Pedro I"	\> "D. Pedro II"\\
	"D. Quixote" \> "Dame à la Rose"\\
	"Damon e Pythias" \> "Danäe"\\
	"Daphne" \> "Das Wellen Opiel"\\ 
	Datas especiais \>  "Davi vencedor de Golias"\\ 
	Demissões e exonerações \> "Derrubador"\\ 
	Desenho \> Desenho à mão livre\\
	Desenho arquitetônico \> Desenho de arquitetura\\
	Desenho de figura	\> Desenho de figura elementar\\
	Desenho de figuras e ornatos \> Desenho de modelo vivo\\
	Desenho de ornatos \> Desenho de ornatos arquitetônicos\\
	Desenho de ornatos e composições \> Desenho de ornatos e figura\\ 
	elementares da arquitetura \> Desenho e decoração de interiores  \\ 
	Desenho e pintura \> Desenho e pintura de paisage, flores e\\ 
	Desenho elementar \> animais\\
	Desenho elementar de ornatos \> Desenho figurado \\		
	Desenho figurado e pintura	\> Desenho geométrico\\
	Desenho geométrico e figurado \> Desenho geométrico e industrial\\
	Desenho geométrico, industrial e de \> Desenho geométrico, plantas e desenho\\
	arquitetura \> topográfico\\ 
	Desenho histórico \>  Desenho industrial\\ 
	Desenho linear \> Desenho topográfico\\ 
	Desenho, geometria, plantas e desenho \> Despesas\\
	topográfico \> Devolução de acervo\\ 
	"Dicionário geográfico do Brasil" \> "Dicionário histórico, geográfico e\\
	"Dicionário universal português ilustrado" \> etnográfico do Brasil" \\	  	   
	Diretório acadêmico \> "Do céu à terra"\\
	Documentos de saúde \> "Documentos, juízo crítico e orçamento\\
	Documentos probatórios civis \> relativo ao monumento patriótico do Brasil"\\
\end{tabbing}

\begin{tabbing}
	\hspace{8,7cm}\=\hspace{1cm}\=\kill
	\textbf{E} \>  \\ 
	"Ebréa"	\> "Efeitos da tempestade"\\
	Eleições \> Elementos de arquitetura\\
	Elementos de arquitetura e desenho de \> "Elementos orgânicos"\\
	ornatos \> "Elevação da cruz"\\ 
	"Eloísa" \>  "Elvia"\\ 
	"Emancipação do elemento servil" \> Embarcações\\ 
	Empréstimo de acervo \> Encadernações\\
	"Ensaio sobre as construções navais \> "Ensaios de pintura" \\	
	indígenas do Brasil"	\> Ensino das Belas Artes\\
	Ensino de Música \> "Epítome de osteologia, miologia e fisiologia\\
	"Ermelinde" \> das paixões"\\
	Escola Alemã \> Escola Italiana\\ 
	Escola Veneziana \>  "Escuela Paleografica"\\ 
	Escravos \> Escultura\\ 
	Escultura de ornatos \> Escultura de ornatos e figura\\
	Esgotos \> "Esmeralda"\\		   	   
	"Estatística do Rio de Janeiro"	\> Estatísticas\\
	"Estátua alegórica do Brasil" \> "Estátua da Ciência"\\
	"Estátua de D. Pedro I" \> "Estátua de D. Pedro II"\\
	"Estátua de João Caetano dos Santos" \> "Estátua de Joaguim Augusto"\\ 
	"Estátua de Miguel Ângelo" \> "Estátua do barão do Rio Branco"\\ 
	"Estátua eqüestre de D. Pedro I" \> "Estátua eqüestre de D. Pedro II"\\ 
	"Estátua eqüestre de duque de Caxias" \> Estatuária\\
	Estatutos \> Estereotomia \\		   	   
	Estética	\> Estética e arqueologia\\
	"Estudo" \> Estudos\\
	"Estudos sobre as pólvoras de guerra \> Etnografia\\
	antigas e modernas" \> "Evangelicoe Historiae"\\ 
	Exames \>  "Exéquias de Camorim"\\ 
	Expedição científica \> Exposição da Companhia Fomentadora das\\ 
	Exposição da Indústria Nacional \> Indústrias e Agricultura de Portugal e suas\\
	Exposição de Chicago \> Exposição de História e Geografia \\	
	Exposição de Produtos Agrículas e \> Exposição Geral de Belas Artes\\ 
	Industriais e de Obras de Arte \> Exposição Geral de Belas Artes em Turim\\
	Exposição Industrial e Artística de Londres \> Exposição Industrial Fluminense \\
	Exposição Internacional de 1862 \> Exposição Internacional de Belas Artes,\\ 
	Exposição Internacional de Córdova \> Santiago do Chile\\
	Exposição Internacional de Filadélfia \> Exposição Internacional de Paris \\		
	Exposição Internacional de produtos	\> Exposição Internacional do Centenário da\\
	naturais, industriais e artísticos em \> Independência do Brasil\\
	Córdova, Argentina \> Exposição Nacional\\ 
	Exposição Nacional de 1866 \> Exposição Nacional na Filadélfia\\
	Exposição Portuguesa no Rio de Janeiro \> Exposição Universal \\		
	Exposição Universal de 1889 em Paris	\> Exposição Universal de Belas Artes em\\
	Exposições de Arte Contemporânea e Arte \> Antuérpia\\
	Retrospectiva \> Exposições\\	   	   	   	   
\end{tabbing}

\begin{tabbing}
	\hspace{8,7cm}\=\hspace{1cm}\=\kill
	\textbf{F} \>  \\ 
	"Faceira"	\> Falecimentos\\
	"Fauno" \> "Favet Neptunus cunti"\\
	Fianças \> Física aplicada\\
	Física e química \> Flauta\\ 
	Flautim \>  "Flora fluminense"\\ 
	"França e Brasil" \> "Francesca de Rimini"\\ 
	Frequência \> "Fugida para o Egito"\\
	Furtos \>  \\		   	   
\end{tabbing}

\begin{tabbing}
	\hspace{8,7cm}\=\hspace{1cm}\=\kill
	\textbf{G} \>  \\ 
	Geometria	\> Geometria descritiva\\
	Geometria descritiva aplicada \> Geometria descritiva e primeiras aplicações\\
	Geometria descritiva, perspectiva e \> às sombreas e à perspectiva\\
	sombras \> "Gioventú"\\ 
	"Glorificação de Anchieta" \>  Grafoestática\\ 
	"Gramática Musical" \> Grande medalha de ouro\\ 
	Gravura \> Gravura de medalhas\\
	Gravura de medalhas e pedras preciosas \> "Grota" \\
	"Guarani" \> "Guja para o ensino do desenho"\\ 
	"Guilherme Tell" \> \\			   	   
\end{tabbing}

\begin{tabbing}
	\hspace{8,7cm}\=\hspace{1cm}\=\kill
	\textbf{H} \>  \\ 
	Habilitação profissional	\> Harmonia\\
	"Hermínia entre os apstores" \> "Hero e Leandro"\\
	Higiene da habitação e saneamento das\> Hinos\\
	cidades \> História\\ 
	História da arquitetura \>  História da arquitetura e legislação especial\\ 
	História das artes plásticas no Brasil \> História das belas artes\\ 
	História das belas artes, estética e \> "Historie de l'art"\\
	arqueologia \> "História e descrição da Real Abadia de \\
	História e teoria da arquitetura \> Altacomba"\\ 
	História natural \> História natural, física e química\\
	História universal \> Histórico escolar\\			   	   
\end{tabbing}

\begin{tabbing}
	\hspace{8,7cm}\=\hspace{1cm}\=\kill
	\textbf{I} \>  \\ 
	"Il costume antico e morderno di tutti i \> "Ilha dos Amores"\\
	popoli" \> Iluminação\\
	"Imagem da Senhora da Conceição"\> "Imagem de São Sebastião"\\
	Imóveis \> Impostos\\ 
	Impressão gráfica \>  "Incredulidade de São Tomé"\\ 
	Informações institucionais \> Inquéritos e processos\\ 
	Inscrições em prédios \> "Instrução Pública no Brasil"\\
	"Instrução Pública" \> "Instruções e programas para os exames \\
	Instrumentos de metal \> de admissão"\\ 
	Inventário de acervo \> \\		   	   
\end{tabbing}

\begin{tabbing}
	\hspace{8,7cm}\=\hspace{1cm}\=\kill
	\textbf{J} \>  \\ 
	"Jesus Cristo em Cafarnaum" \> "Judas"\\
	"Juramento da Regência Trina \> Júri\\
	Permanente"\> \\	   	   
\end{tabbing}

\begin{tabbing}
	\hspace{8,7cm}\=\hspace{1cm}\=\kill
	\textbf{L} \>  \\ 
	"L' Itália" \> "L'art ennoblit"\\
	"La Bohèmme" \> "La Republica Argentina en su primer\\
	"La sorte é gittada nel grembo; ma nel\> Centenario"\\
	signore procede tutto il giudizio di essa". \> "Lanzas"\\ 
	"Laocoonte" \>  "Le beau, le vrai, l'utile"\\ 
	"Le Fabriche (?) de Miceli Sanmicheli" \> Legislação da construção e economia\\ 
	Legislação  \> política\\
	"Legislação das construções" \> "Les Galeries Publiques de l'Europe"\\
	Licenças \> "Lindóia"\\
	Litografia\> Livre docência\\
	"Lucrécia" \> "Luís de Camões"\\
\end{tabbing}

\begin{tabbing}
	\hspace{8,7cm}\=\hspace{1cm}\=\kill
	\textbf{M} \>  \\ 
	"Madona"	\> "Mahoslixe"\\
	"Manhã azul" \> "Mapa arquitetural"\\
	"Mapa escolar do Brasil" \> "Marabá"\\
	"Martírio de São Januário" \> Matemática\\ 
	Matemática complementar \>  Matemática superior\\ 
	Matemáticas \> Matemáticas aplicadas\\ 
	Matemáticas aplicadas às artes \> Matemáticas e ciências físicas\\
	"Matemáticas elementares" \> Matemáticas elementares \\	
	Materiais de construção	\> Materiais de construção e sua resistência\\
	Materiais permanente e de consumo \> Material didático\\
	"Maternidade" \> Matrículas\\
	Mausoléus \> Mecânica\\ 
	Mecânica e resistência de materiais \>  "Mecanismos e proporções da figura\\ 
	Medalha de bronze \> humana"\\ 
	Medalha de ouro \> Medalha de prata\\
	Medalhas \> Medalhas comemorativas\\		   	   
	"Meditação"	\> Membros correspondentes\\
	Membros honorários \> Menção honrosa\\
	Menção honrosa de $1^a$ classe \> Menção honrosa de $1^o$ grau\\
	Menção honrosa de $2^a$ classe \> "Mendonza"\\ 
	"Mercúrio" \> "Mestre Aleijadinho"\\ 
	"Missa de Bolcena" \> "Missal"\\ 
	Mitologia \> Modelagem\\
	Modelagem de ornatos e elementos de \> Modelo vivo \\		   	   
	arquitetura	\> Moedas\\
	"Moema morta" \> "Moisés salvado das águas"\\
	Moldagem \> "Monólogo"\\
	"Monumento a d. Pedro I" \> "Monumento a D. Pedro IV"\\ 
	"Monumento à descoberta do Brasil" \>  "Monumento aos Andradas"\\ 
	Monumentos \> "Morte de filósofo"\\ 
	"Museu Francês" \> Música\\
	Música vocal \>  \\	   	   	   	   
\end{tabbing}

\begin{tabbing}
	\hspace{8,7cm}\=\hspace{1cm}\=\kill
	\textbf{N} \>  \\ 
	"Narração de Filetos" \> "Natalis Evangelicae Historiae Imagines"\\
	Noções concretas de história natural, física \> Noções de canto\\
	e química aplicadas às Belas Artes\> "Noções elementares de Arqueologia"\\
	Normas e instruções \> "Nossa Senhora do Rosário"\\ 
	"Nu deitado" \>  "Nu masculino sentado"\\ 
	"Nu" \> \\ 
\end{tabbing}

\begin{tabbing}
	\hspace{8,7cm}\=\hspace{1cm}\=\kill
	\textbf{O} \>  \\ 
	"O africano" \> "O Aleijadinho em Vila Rica"\\
	"O banho" \> "O casamento de Sua Majestade o\\
	"O cavalão"\> Imperador, em Nápoles em 1843"\\
	"O dueto" \> "O esplendor do amor (The sunshine of\\ 
	"O estudo e a Renommé" \>  love)"\\ 
	"O grito do Ipirana" \> "O Guarani"\\ 
	"O marquês de Herval nos campos do  \> "O óbulo da viúva"\\
	Paraguai" \> "O rapto de Helena"\\
	"O Real Museu de Turim" \> "O Sacramento da Extrema-Unção"\\
	"O sermão da montanha" \> "O sono"\\
	"O último tamoio" \> Obras e edificações\\
	Óperas \> Orçamentos\\
	"Organização das ordens honoríficas do \> Ornatos e decorações\\
	Império do Brasil"\> Ornatos e figuras\\
	Osteologia, miologia e fisiologia das \> \\
	paixões \> \\
\end{tabbing}

\begin{tabbing}
	\hspace{8,7cm}\=\hspace{1cm}\=\kill
	\textbf{P} \>  \\ 
	Paisagem \> Paisagem histórica\\
	Paisagem, flores e animais \> Paisagens, flores e animais\\
	"Palácio para S. A. Imperial"\> "Palmam qui meruit ferat"\\
	"Panorama do Rio de Janeiro" \> Pareceres\\ 
	"Passagem de Humaitá" \>  Passes em transporte\\ 
	Patentes industriais \> "Pedreira de São Lourenço"\\ 
	"Penture mate"  \> "Pensativo"\\
	Pensionistas \> Pensões\\
	Pequena medalha de ouro \> Pequena medalha de prata\\
	"Pequeno napolitano" \> Perspectiva\\
	Perspectiva e sombras \> Perspectiva e teoria das sombras\\
	"Pesca" \> Piano\\
	"Pigmaleão" \> Pintura\\
	Pintura cenográfica \> Pintura de aquarela\\
	Pintura de paisagem \> Pintura de paisagem, flores e animais\\
	Pintura histórica \> Pintura, escultura e gravura\\
	"Porto do Rio de Janeiro" \> "Praia da Boa Viagem"\\
	"Prece" \> Premiações\\
	Prêmio Beethoven\> Prêmio Caminhoá\\
	Prêmio Cocural \> Prêmio Condessa d'Eu\\ 
	Prêmio de $1^a$ ordem\> Prêmio de $2^a$ ordem\\ 
	Prêmio de viagem \> Prêmio em dinheiro\\
	Prêmio Imperatriz do Brasil \> Prêmio Imprensa\\
	Prêmio Princesa Imperial \> "Primeira comunhão na América"\\
	Professorado de desenho \> Publicação de atos e documentos\\
	Publicação de obras \> \\
\end{tabbing}

\begin{tabbing}
	\hspace{8,7cm}\=\hspace{1cm}\=\kill
	\textbf{Q} \>  \\ 
	Quadro funcional \> Questões disciplinares\\
	Química orgânica \> \\
\end{tabbing}

\begin{tabbing}
	\hspace{8,7cm}\=\hspace{1cm}\=\kill
	\textbf{R} \>  \\ 
	Rabeca \> Rabeca surda\\
	"Rebeca" \> Recolhimento de documentos\\
	Registro de correspondência\> Regras de acompanhar e órgão\\
	Regras de harmonia e de harmonia  \> "Regulamento das Exposições Gerais\\ 
	acompanhamento práticos \>  de Belas Artes"\\ 
	"Relatório apresentado a Sua Majestade \> "Relatório e trabalhos estatísticos do ano de\\ 
	Imperial pelo presidente da Comissão  \> 1873"\\
	Brasileira junto \> Relatórios\\
	"Relatórios da Comissão encarregada de \> Remessas de livros e documentos\\
	tratar do monumento a D. Pedro IV de \> Remessas de obras de arte\\
	Portugal" \> "Rendição de Brida"\\
	Reproduções de obras de arte \> Requinta\\
	Resistência aos materiais \> "Retrato da baronesa de Araguaia"\\
	"Retrato da Exma. Sra. A. A." \> "Retrato de Alfredo Galvão"\\
	"Retrato de D. João VI menino" \> "Retrato de D. Pedro I"\\
	"Retrato de D. Pedro II" \> "Retrato de D. Pedro II"\\
	"Retrato de D. Teresa Cristina" \> "Retrato de Debret"\\
	"Retrato de Domingos Antônio de Siqueira" \> "Retrato de Felipe Lopes Neto"\\
	"Retrato de Grandjean de Montigny"\> "Retrato de João Batista Debret"\\
	"Retrato de José Joaquim de Andrade \> "Retrato de José Ribeiro de Sousa Fontes"\\ 
	Neves"\> "Retrato de Lagrange"\\ 
	"Retrato de Manuel da Silva e suas \> "Retrato de Manuel de Araújo Porto Alegre\\
	enteadas" \> "Retrato de Manuel Luís Osório"\\
	"Retrato de menina" \> "Retrato de S. M. o Imperador"\\
	"Retrato do barão de Araguaia" \> "Retrato do barão de Santo Ângelo"\\
	"Retrato do conde do Rio Pardo" \> "Retrato do conde João Maurício de\\
	"Retrato do escritor Gonzaga Duque" \> Nassau Siegen"\\
	"Retrato do general Lima e Silva" \> "Retrato do Infante D. João VI"\\
	"Retrato do marquês de Inhambupe" \> "Retrato do marquês de Sapucaí"\\
	"Retrato do visconde do Bom Retiro"\> Retratos\\
	Reuniões de serviço \> "Rio Capibaribe"\\
	"Roma" \> "Rua Augusta - Parati"\\
	Rudimentos" \> Rudimentos de música, solfejo coletivo e\\
	Rudimentos de música, solfejo coletivo e\> individual e noções gerais de canto para o\\
	individual e noções gerais de canto para o \> sexo\\
	sexo \> Rudimentos e solfejos\\
	Rudimentos e solfejos para o sexo feminino \> Rudimentos e solfejos para o sexo masculino\\
	Rudimentos, solfejos e noções gerais de canto \> \\
\end{tabbing}

\begin{tabbing}
	\hspace{8,7cm}\=\hspace{1cm}\=\kill
	\textbf{S} \>  \\ 
	Salão de Paris \> "Santo Estêvão"\\
	"São Francisco" \> "São João Batista"\\
	"São Paulo"\> "São Sebastião"\\
	"Saudade" \> Segurança patrimonial\\ 
	Selos \>  "Sertório com sua corça"\\ 
	Serviços militares \> Sinistros\\ 
	Sistema e detalhes de construção  \> "Sócrates afastando Alcebíades do vício"\\
	Solfejo \> Solfejo e canto\\
	Solfejo e noções gerais de canto para as \> Sombras e perspectiva\\
	alunas do $2^o$ e $3^o$ ano \> Substituição de professor\\
\end{tabbing}

\begin{tabbing}
	\hspace{8,7cm}\=\hspace{1cm}\=\kill
	\textbf{T} \>  \\ 
	"Tarantela" \> Tecnologia das profissões elementares\\
	"Telêmaco de volta à ilha de Itaca..." \> Tempo de serviço\\
	"Tentação de Santo Antão"\> Teoria\\
	Teoria da arquitetura \> Teoria e filosofia da arquitetura\\ 
	Teoria e piano \>  Teoria musical\\ 
	"Theatrum" \> "To be or not to be"\\ 
	"Tomada do forte do Itapiru"  \> Topogragia\\
	Topografia e desenho topográfico \> "Tous les arts sont frères"\\
	Trabalhos gráficos \> Transferência de acervo\\
	Trigonometria \> "Triunfo de Netuno"\\
	Trompa \> Trompa e outros instrumentos de metal\\
	"Tronco de Pagnart" \> \\
\end{tabbing}

\begin{tabbing}
	\hspace{8,7cm}\=\hspace{1cm}\=\kill
	\textbf{U} \>  \\ 
	"Último momento da morte de Sócrates" \> "Uma fiandeira"\\
	Urbanismo \> Urbanismo - arquitetura paisagista\\
	"Usos e costumes de todos os povos do\> Urbanização\\
	universo" \> \\
\end{tabbing}

\begin{tabbing}
	\hspace{8,7cm}\=\hspace{1cm}\=\kill
	\textbf{V} \>  \\ 
    Vacância de cargo \> "Vallée de Saint Valmeront-Auvergne"\\
	"Vaso com flores secas" \> Vencimentos\\
	"Vênus Calipígia"\> "Vênus de Médicis"\\
	"Ver non semper floret" \> Vestimentas \\
	Viagem de estudos \> Violino\\
	Violoncelo \> Violoncelo e contrabaixo\\
	"Visita de Baco ao palácio de Netuno"\> Visitas\\
	"Voltaire abençoando o neto de Franklin em \> "Vista de São Pedro de Roma"\\
	nome de Deus e da liberdade" \> \\
\end{tabbing}

\begin{tabbing}
	\hspace{8,7cm}\=\hspace{1cm}\=\kill
	\textbf{X} \>  \\ 
	Xilografia \> Xilogravura\\
\end{tabbing}